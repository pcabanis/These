%!TEX root = Manuscrit.tex
\chapter{Introduction}
\label{chap:intro}

%\localtableofcontents
\minitoc
\section{Le myocarde}

\subsection{Fonctionnement et anatomie du cœur }

Le cœur est un organe central pour le fonctionnement du vivant. Il permet d’amener le sang riche en oxygène dans tous les organes du corps, puis de rapporter le sang appauvri en sortie des organes dans les poumons pour recharger le sang en oxygène. Les facteurs extérieurs et génétiques influent sur le déroulement de ce cycle cardiaque et vont avoir des conséquences désastreuses pouvant entrainer de graves problèmes médicaux. \cite{Miles2019} 

Pour irriguer l’ensemble du corps humain, le cœur est connecté à un réseau de vaisseaux sanguins allant du cœur aux organes (les artères) puis des organes au cœur (les veines)  et pour finir, dans les organes, de petits vaisseaux pour irriguer l’ensemble de l’organe (les capillaires). Le cœur est lui-même irrigué par un réseau de vaisseaux sanguins, appelés les coronaires. 

Le cœur est constitué de quatre cavités : l’oreillette gauche (OG), l’oreillette droite (OD) les ventricules gauche (VG) et droite (VD). Il est divisé schématiquement en deux cœurs, le droit (OD-VD) et le gauche (OG-VG). Le septum interauriculaire sépare les deux oreillettes et le septum intraventriculaire (IVS) sépare les deux ventricules. Le passage de l’oreillette au ventricule se fait par l’intermédiaire d’une valve, tricuspide à droite et mitrale à gauche. 

Le cœur est lui-même entouré d’une membrane appelé péricarde permet le maintien en place du cœur et protégeant le cœur d’éventuelles infection du thorax.


\begin{figure}[!htbp]
  \begin{center}
    \includegraphics[width=0.9\textwidth]{Chapitre1/description_coeur.png}
  \end{center}
  \caption{Anatomie du cœur. Extrait de https://www.fedecardio.org/je-m-informe/le-fonctionnement-du-coeur/}
  \label{fig:description_coeur}
\end{figure}

Le sang circule dans un sens unique et synchrone, le cycle cardiaque. Le cœur gauche propulse le sang riche en oxygène dans les zones supérieures et inférieures du corps humain après que le sang ait été enrichi en oxygène dans les poumons grâce aux contractions du cœur droit. Le cœur est situé à proximité des poumons, de ce fait le cœur droit est moins musclé que le cœur gauche qui lui doit permettre l’irrigation en sang riche en oxygène dans tout l’organisme. Le cycle cardiaque est divisé en deux parties, les contractions (systole) et relaxations (diastole), répétable à l’infini. 

Les contractions du cœur sont rendues possibles grâce à la capacité du tissu cardiaque  (appelé myocarde) à se contracté de façon séquentielle. 

\subsection{La paroi du cœur}
La paroi du cœur est constituée de trois couches, de l’extérieur vers l’intérieur, l’épicarde, le myocarde et l’endocarde. Chaque couche a un rôle unique dans la contraction du cœur. 

L’épicarde est la couche extérieure du cœur (le mésothéliome), c’est une membrane composée de cellule épithéliales pavimenteuses protégeant le reste de la paroi, il est attaché au péricarde.

A l’inverse, l’endocarde tapisse les cavités cardiaques. IL est composé d'un épithélium pavimenteux simple, l'endothélium, qui est en continuité avec celui des vaisseaux sanguins. L'endocarde est relié au myocarde par une couche sous-endocardique de tissu conjonctif lâche, richement vascularisée, contenant des fibres nerveuses et les ramifications du réseau de Purkinje au niveau des ventricules.

Le myocarde est la partie centrale de la paroi du cœur, c’est la couche la plus épaisse permettant la contraction des parois. C’est un tissu hétérogène constitué de cellules cardiaques, les myocytes ou cardiomyocytes séparées les unes des autres par la matrice extra-cellulaire (collagène, MEC). Les myocytes se connectent entre elles sous la forme de chaînes/agrégats de myocytes formant des fibres musculaires, les myofibres \footnote{Le terme myofibre est soumis à débats car le mot « fibre » renvoi à la fibre squeleto-musculaire, différente de la réalité histologique des fibres cardiaques. }. Le couple cardiomyocyte- MEC constitue 70  \% du volume du myocarde. 

Les myocytes sont de formes allongés/cylindriques de 100 à 300 $\mu$m de long pour 20 à 30 $\mu$m de diamètre. La membrane des myocytes est une barrière semi-perméable (constitué à 70  \% de lipides) entre les milieux aqueux intra et extra cellulaire.

Il existe d’autres cellules que les cardiomyocytes dans le myocarde : 

 \begin{bulletList}
 \item Cellules musculaires lisses et endothéliales composant la paroi des coronaires
 \item Cellules nerveuses permettant la régulation du cœur par le système nerveux autonome
 \item Fibroblastes participant à la formation du collagène de la MEC nécessaire à la flexibilté et solidité de la paroi cardiaque \cite{BAYOMY2012823}
 \item cellules conductrices faisant partie du système de conduction du cœur (cellules de Purkinje …)
\end{bulletList}



Des cellules adipeuses (graisse) existent aussi dans l’épicarde et autour de certains vaisseaux, elles ont un rôle de protection de l’organisme mais une trop grande prolifération et/ou un remplacement des cardiomyocytes de ces cellules peut entrainer des dysfonctions cardiaques. \cite{samanta_role_2016} \cite{hatem_epicardial_2014}


\subsection{Organisation du myocarde : Chaines de myocytes et orthotropie}
Dans la partie précédente, nous avons vu que le myocarde était constitué en majorité de cellules de forme cylindrique appelés cardiomyocytes agrégés sous la forme de chaînes. 

Un modèle théorique représente les myofibres comme étant un ensemble, le cœur pouvant être déroulé et enroulé sous la fome d’une bande, la bande myocardique, théorisé par Torrens-gasp ne 1957 \cite{Kocica2006}. Ce modèle se base sur une dissection du cœur par un opérateur (Figure \ref{fig:model_fibre}.A).
 Ce modèle est remis en cause depuis le début des années 2000 avec l’apparition de nouvelles méthodes d’imagerie permettant d’obtenir l’orientation des myofibres en trois dimensions \cite{MacIver2017_end_I} \cite{MacIver2017_end_II}.  

\begin{figure}[!htbp]
  \begin{center}
    \includegraphics[width=0.9\textwidth]{Chapitre1/figure_vg_helix_laminaire.png}
  \end{center}
  \caption{Descrption de l'orientation des myofibres dans le coeur. A : Modèle de la bande myocardique par Torrens-gasp. B Architecture laminaire du myocarde en contration et en dilatation \cite{NiellesVallespin2019}, plans laminaires \cite{Khalique_2020}}
  \label{fig:model_fibre}
\end{figure}

D’un point de vue macroscopique, les myofibres ne sont pas orientées dans un sens unique (un seul faisceau) comme dans les fibres musculaires. Elles sont parallèles à la surface à l’épicarde mais varie linéairement (-60° à 60 °) et hélicoïdale de manière transmurale dans le VG (Figure \ref{fig:model_fibre}.B), appelé règle de l’helix angle \cite{NiellesVallespin2019} \cite{STREETER1969}. Les fibres cardiaques sont ensuite regroupées sous la forme de feuillets de quatre à six cellules d’épaisseur Figure \ref{fig:model_fibre}.C). Dans le VD, l’architecture des myocytes est plus complexe \cite{Vetter2005} avec des changements abrupts d’orientation, notamment dans les voies d’éjections du ventricule droit (RVOT). 

L’architecture du myocarde est complexe mais structurée appelé structure orthotrope \cite{NiellesVallespin2019} qui est définie selon trois axes : 

\begin{bulletList}
 \item L’axe de la myofibre
 \item L’axe du plan du feuillet (orthogonale à l’axe de la myofibre)
 \item L’axe orthogonale du plan du feuillet
\end{bulletList}


L’épaisseur des parois du VG et VD ne sont pas égales malgré que le VG et VD pompent la même quantité de sang, la paroi du VG est plus épaisse ( trois fois supérieur) pour faire face à la résistance du système vasculaire entier.

\subsection{Cycle cardiaque et contraction des cardiomyocytes}

La circulation du sang dans le cœur suit un schéma prédéfini permettant au sang pauvre en oxygène de s’enrichir dans les poumons puis d’alimenter l’organisme en oxygène (Figure \ref{fig:cycle_sang}).

\begin{figure}[!htbp]
  \begin{center}
    \includegraphics[width=0.9\textwidth]{Chapitre1/cycle_sang.png}
  \end{center}
  \caption{Circulation du sang dans le cœur. Le sang pauvre en oxygène arrive depuis les parties inférieurs et supérieurs du corps dans l’OD (2) par les veines caves inférieurs et supérieurs (1). Le sang est ensuite déversé dans le VD (3) au travers de la valve tricuspide puis en se contractant, le VD envoi le sang dans les poumons (5) au travers des artères pulmonaires (4). En parallèle, le sang riche en oxygène arrive des poumons (6) dans OG (7) puis est déversé dans le VG (8) par la valve mitrale. La contraction du VG envoi le sang dans l’aorte (9) puis le sang est distribué dans tout l’organisme (9). https://www.chuv.ch/fileadmin/sites/car/images/coeur2.png}
  \label{fig:cycle_sang}
\end{figure}

Les contractions du VG et du VD sont rendues possible grâce à la capacité des myocytes de se contracter sous l’effet d’un potentiel électrique \cite{renard:tel-04344684}. On appelle potentiel d’action (PA) cardiaque la modification de la différence de potentiel entre l’intra et l’extra-cellulaire par les échanges entre différents ions : potassium (K+), sodium (Na+) et calcium (Ca2+) . 
\\
Le PA peut être résumé en plusieurs phases comme suit  (Figure \ref{fig:fig_PA}.A): 
\begin{enumerate}
\setcounter{enumi}{-1}
\item Dépolarisation rapide
\item Repolarisation rapide
\item Phase de plateau
\item Repolarisation lente
\item Dépolarisation diastolique spontané

\end{enumerate}

\begin{figure}[!htbp]
  \begin{center}
    \includegraphics[width=0.9\textwidth]{Chapitre1/figure_PA.png}
     \end{center}
    \caption{PA et propagation du PA. A : Définition du potentiel d'action avec les cinq phases qui le consititue et les échanges ioniques. B : Propagation du PA depuis le noeud sinusale jusque dans les ventricles. L'ECG de surface est la combinaison des PA locaux}
  \label{fig:fig_PA}
\end{figure}

La phase de dépolarisation rapide est induite par de faibles courants électriques venant du nœud sinusale (situé dans l’OD). Les cellules dans le nœud sinusale disposent de leurs propres automaticité (capacité à générer une différence de potentiel) à une fréquence régulière (70 fois par minute). La tension se propage ensuite dans tout le cœur (oreillettes puis ventricules) suivant un chemin de conduction permettant à la tension d’aller de l’OD à l’apex du cœur. Il existe alors une différence temporelle et spatiale entre l’activation des différents myocytes (Figure \ref{fig:fig_PA}.B).

Le cycle cardiaque est donc divisé en quatre phases : i) contraction des oreillettes pour permettre au sang d’aller dans les ventricules, les valves mitrale et tricuspide s’ouvrent. ii) Contractions des ventricules, fermeture des valves mitrale et tricuspide et ouverture des valves aortique et pulmonaires, le sang quitte les ventricules pour aller dans l’organisme ou les poumons, c’est la systole.  iii) Après la systole, les ventricules se relâchent et le sang afflue dans les oreillettes, c’est la diastole. iv)  Les oreillettes se contractent légèrement pour avoir un maximum de sang dans les oreillettes et le cycle suivant commence.

\subsection{Orientation chaines de myocytes et PA}

Le paragraphe précédent montre comment un cardiomyocyte se contracte sous l’effet d’un PA et comment le cœur dans son ensemble réagit à ces contractions microscopiques. Or, le myocarde est organisé selon une architecture complexe mais structurée. Les myocytes sont connectés entre elles dans le sens longitudinal par les disques intercalaires (contenant des jonctions communicantes ou jonction gap, permettant le transfert du PA). Les myocytes ont une géométrie longitudinale, le passage d’une cellule à l’autre est plus dans ce sens que dans le sens transverse car le PA est moins interrompu moins fréquemment que dans le sens transversale (plus de membranes cellulaires à traverser, Figure \ref{fig:fig_fibre_et_PA}). 

L’orthotropie du tissu cardiaque influe aussi sur la propagation du PA, il existe aussi une conduction dans le sens du plan et une autre dans entre les plans laminaires. Le rapport entre la vitesse de conduction dans le sens des myofibres, vitesse de conduction dans le plan et vitesse de conduction entre les plans est approximativement de 4/2/1 \cite{Caldwell2009} \cite{Hooks2007}. 

\begin{figure}[!htbp]
  \begin{center}
    \includegraphics[width=0.9\textwidth]{Chapitre1/fibre_et_PA.png}
  \end{center}
  \caption{Anisotropie du tissu cardiaque et son influence sur la propagation du PA. La conduction longitudinale (flèche orange) dans la longueur des cardiomyocytes résulte d’une conduction cytoplasmique rapide à basse résistance (flèches vertes) interrompue de manière peu fréquente par une résistance de couplage intercellulaire qui ralentit la conduction (flèches bleues). La CV transversale (flèche violette), dans la largeur des cardiomyocytes, est quant à elle plus faible car elle est interrompue plus fréquemment par la résistance de couplage intercellulaire, d’autant plus que la résistance de couplage intercellulaire est supérieure au niveau de la membrane latérale (ML) des cardiomyocytes en raison du regroupement des jonctions communicantes au niveau des disques intercalaire (DI). Enfin, la faible CV transversale est également due à une répartition préférentielle des canaux NaV1.5 dans les DI qui par conséquent réduit le courant INa et donc la Vmax observée à la ML. Cette hétérogénéité de conduction selon l’axe de propagation correspond à l’anisotropie de la conduction cardiaque. extrait de la thèse du Dr E.Renard avec permission \cite{renard:tel-04344684}}
  \label{fig:fig_fibre_et_PA}
\end{figure}

\clearpage

Jusqu’ici, nous ne sommes occupés que des cardiomyocytes et leurs impacts sur la propagation du PA, or le myocarde est un tissu biologique hétérogène. Les phases de polarisation et dépolarisation des myocytes suivent un cycle régulier espacé par un temps de repos entre deux cycles.

La MEC est un élément peu électro-conductif/isolant (collagène) \cite{Zannad2005}, expliquant que la propagation du PA soit freinée dans le sens orthogonal au plan laminaire. Dans le cas d’un cœur sain, l’influence de la MEC sur le PA est normale, c’est-à-dire qu’elle ne perturbe pas le fonctionnement classique du cœur.  

Par contre, si un obstacle local venait à se former (mort cellulaire et remplacement des myocytes par la MEC ou par des cellules adipeuses), la propagation du PA serait ralentie, voir bloquée. Dans ce cas, il est possible que les chemins de conductions soient perturbés entraînant alors de nouveaux chemins qui amèneront deux ondes électriques au même cardiomyocyte. Le temps de repos entre du PA n’est plus respecté, alors la fonction contractile de la cellule est perturbée et par extension c’est le fonctionnement entier du cœur qui est dégradé. On appelle ces troubles du rythmes cardiaques les arythmies, elles sont une famille des maladies cardiaques. Le triangle de Coumel (Figure \ref{fig:fig_coumel}) est une représentation schématique des causes des arythmies, il est une analogie avec le triangle du feu/des incendies avec le substrat (le tissu cardiaque dégradé / du bois sec), des modulateurs qui vont favoriser ou non l’arythmie (maladies génétiques, pratiques sportives…/vent, temps sec…) et une gâchette de déclenchement ou trigger (phénomène de rentrée /allumette). Le myocarde peut aussi se dilater (dilation) ou grossier (hypertrophie) à cause de facteurs génétiques ou extérieurs au cœur (obésité, sport intense…). 

\begin{figure}[!htbp]
  \begin{center}
    \includegraphics[width=0.9\textwidth]{Chapitre1/triangle_coumel.png}
     \end{center}
    \caption{Triangle de Coumel : Les arythmies sont la combinaison de plusieurs facteurs entrainement un changement brutal du rythme cardiaque (accélération : tachycardie ou ralentissement : bradycardie)}
  \label{fig:fig_coumel}
\end{figure}

Pour conclure, le cœur est un organe complexe dont la fonction première est de permettre de transporter l’oxygène et les nutriments dans tous les organes. Il possède une architecture ordonnée mais hétérogène entre les différentes zones qui le constitue, chaque zone ayant une fonction différente des autres. Le myocarde (tissu de la paroi du cœur) est constitué de cellules contractiles (les myocytes) qui sous l’impulsion d’un potentiel d’action vont se contracter et la contraction de l’ensemble des myocytes va permettre le transport du sang dans le cœur et l’organisme. Une vue macroscopique du myocarde révèle une architecture orthotrope avec les myocytes agrégés sous la forme de chaines, ces chaines formant ensuite des feuillets séparés entre eux par du collagène. Les contractions du cœur (systole) vont venir espacées ces feuillets et la phase de repos (diastole), pendant que le sang rempli la cavité ventriculaire ou auriculaire, va rapprocher les feuillets. Des facteurs extérieurs peuvent dégrader le myocarde et perturber ce cycle vertueux, c’est pourquoi il est nécessaire d’étudier la structure du myocarde pour prévenir d’éventuelles maladies cardiaques. 



\section{Les bases de l'IRM}
\subsection{Du proton à l’image}
\subsubsection{La physique, du proton au signal}

L’imagerie par résonnance magnétique (IRM) est une technologie d’acquisition d’image en trois dimensions non invasives utilisant les propriétés gyromagnétiques de différents éléments chimiques tels que l’hydrogène, le phosphore ou le carbone. Comme le corps humain est constitué majoritairement de molécules d’eau $H_2O$ ( environ 60 $\%$ pour un homme adulte), il est courant que les acquisitions IRM soient centrées sur la fréquence de résonnance de l’hydrogène. Ces acquisitions reposent sur le phénomène de Résonance Magnétique Nucléaire (RMN) qui se déroule en trois étapes : la polarisation, la résonnance et la relaxation \cite{HolgerFrsterling2009}. 

La polarisation est l’étape préliminaire avant toute acquisition, elle s’opère lorsque l’objet à imager ( patient/échantillon) est positionné dans le tunnel de l’imageur IRM. Le proton (hydrogène) possède des propriétés magnétiques qui peuvent être représenté sous la forme d’un vecteur en rotation sur lui-même : le spin. Lorsque le proton n’est soumis à aucun champ magnétique, le vecteur aura une orientation aléatoire, cependant, s’il est soumis à un champ magnétique suffisamment fort, il sera orienté dans l’axe du champ magnétique. Les spins du protons ou spins protoniques ont toujours leurs moments propres gyromagnétiques qui précesse (ou plus sommairement, tourne) à la fréquence de Larmor $\nu 0 =\gamma H \cdot B_0 $ avec $\gamma H$ le rapport gyromagnétique du proton et $B_0$ le champ magnétique de l’imageur. La polarisation est donc l’aimantation M induite par l’interaction des spins et de $B_0$. Si nous décomposons M selon les axes x,y et z, nous avons donc $M_{x,y} = 0$ et $M_z$ = $M_0$, avec $M_0$ l’aimatation totale. 

La deuxième étape est la résonnance. Avant toute chose, il est nécessaire de comprendre que l’aimantation M demeure trop faible  par rapport à $B_0$ pour être mesurer en l’état. Pour cela nous allons rajouter une antenne d’émission (Tx) qui va venir induire un nouveau champ B1 orthogonal à $B_0$ et en rotation autour de $B_0$ à la fréquence de Larmor (dépendante de la valeur du champ $B_0$).  L’application d’onde radiofréquence (RF) va faire basculer l’aimantation M vers le champ $B_1$. L’aimantation M n’est donc plus parallèle à l’axe z et cela engendre donc une aimantation $M_{x,y}$ (aimantation transversale) en plus de l’aimantation $M_z$ (aimantation longitudinale), dépendantes de l’angle de bascule $\alpha$ (\textit{flip angle}) appliqué à M :
\begin{equation}
\nonumber
	M_{x,y} = M0 \cdot \sin{\alpha}
\end{equation}

\begin{equation}
\nonumber
	M_z =\ M0 \cdot \cos{\alpha}
\end{equation}

Avec $\alpha =\ \gamma H\cdot \ \tau$ et $\tau\ $ durée de l’impulsion RF \cite{Ernst1966}.

Lorsque l’impulsion RF n’est plus maintenue, le proton cherche à revenir à sa position d’équilibre, c’est-à-dire revenir parallèle à l’axe de  $B_0$, c’est la phase de relaxation. Le signal qui résulte du retour à l’équilibre est appelé le signal de décroissance d’induction libre plus communément appelé FID (\textit{Free Induction Decay}). 
\\
Comme M est divisée en deux composantes $M_{x,y}$  et $M_z$, le retour à l’équilibre sera caractérisé par le retour de l’aimantation transverse $M_{x,y}$  à 0 et un retour de l’aimantation longitudinale $M_z$ à sa valeur initiale en suivant les équations de Bloch \cite{HolgerFrsterling2009}: 
\begin{equation}
\nonumber
	M_{x,y}\left(t\right)=M_{x,y}\left(0\right)\ {\cdot\ e}^{-\frac{t}{T2}}\cdot\ e^{-i\omega_0t}	
	\end{equation}

\begin{equation}
\nonumber
	M_z\left(t\right)=M_z\left(0\right)\ {\cdot(1-\ e}^{-\frac{t}{T1}})
	\end{equation}

T1 et T2 sont les constantes de temps caractéristiques de la relaxation en RMN, respectivement 63 $\%$ du temps nécessaire à l’aimantation longitudinale à revenir à sa valeur initiale $M_z$(t = 0) et 63 $\%$ du temps nécessaire à l’aimantation transverse à revenir à 0. La valeur de T1 est dépendante du tissu et de sa structure (taille…). La relaxation T1 correspond à l’interaction spin-réseau, c’est-à-dire entre le proton et le champ $B_0$ alors que la relaxation T2 correspond à l’interaction spin-spin, l’interaction entre les protons.

Concernant le T2, si l’on regarde de plus près les équations de Bloch, l’aimantation $M_z$ ne dépend d’aucuns paramètres physiques, là ou Mx,y dépend de la fréquence de Larmor dépendante elle-même de la valeur du champ  $B_0$. Les inhomogénéités de $B_0$ induisent donc une accélération de la décroissance de $M_{x,y}$  résultant une nouvelle constante de temps T2* plus courte que T2 car les spins ne précessent plus à la même fréquence. 

\subsubsection{Transformée de Fourier, espace de Fourier et transformée inverse}

Un signal est caractérisé par trois éléments : la fréquence, l’amplitude et la phase. L’analyse fréquentielle utilise la Transformée de Fourier (TF)  pour analyser ces signaux et exprimer l’amplitude (et la phase) en fonction de la fréquence. 
 La TF d’une fonction f(t) en 1 D est :
 
 \begin{equation}
 \nonumber
  TF(f(t)) = \int_{-\infty}^{+\infty}{f(t)\cdot e^{-j2\pi \nu t}dt}
 \end{equation}
 avec $\nu$ la fréquence (Hz).  
La TF d’un signal réel est un signal complexe avec une phase et une magnitude, toutes deux exprimé en fonction de la fréquence. Chaque fréquence composant un signal 1D dans le domaine temporel est représenté sous la forme d’une impulsion de Dirac dans le domaine fréquentiel à cette même fréquence (Figure \ref{fig:ft_1D}).  

\begin{figure}[!htbp]
  \begin{center}
    \includegraphics[width=0.9\textwidth]{Chapitre1/1D_fourier.png}
     \end{center}
    \caption{Trois signaux temporels sinusoïdales avec des amplitudes et des fréquences différentes (gauche) et leurs TF (droite). L'amplitude du temporel influe sur l'ampltude de la  TF et la fréquence du signal temporel décale le pic de la TF vers 0 si la fréquence diminue et vers l'infini si la fréquence augmente}
  \label{fig:ft_1D}
\end{figure}
\clearpage
Dans le cas d’un signal en 2 dimensions  (image),  la TF sera aussi un signal complexe en 2 dimensions (Figure \ref{fig:ft_2D}). Le plus souvent, ce signal sera exprimé spatialement en fonction des axes x et y, sous la forme f(x,y) avec sa transformée de Fourier F associé définie avec les fréquences spatiales kx et ky : 
\begin{equation}
\nonumber
F(kx,ky)=\iint_{-\infty}^{+\infty}{f(x,y)\cdot e^{-j2\pi(kx\cdot x+ky\cdot y)}dxdy}
\end{equation}
Un espace de Fourier en 2D est aussi appelé espace K, avec en son centre (kx = ky = 0 $m^{-1}$ ) les basses fréquences, les hautes fréquences étant les points éloignés du centre.

\begin{figure}[!htbp]
  \begin{center}
    \includegraphics[width=0.9\textwidth]{Chapitre1/2D-ft.png}
     \end{center}
    \caption{Espace de Fourier et Espace Image. Gauche : Image en nuance de gris dans le domaine spatial, Droite : Espace de Fourier de l’image de gauche calculé à partir de la Transformée de Fourier.}
  \label{fig:ft_2D}
\end{figure}



Il existe aussi la transformée inverse de Fourier (TFi) qui permet de passer du domaine fréquentiel au domaine spatial :
\begin{equation}
\nonumber  
f(x,y)=\iint_{-\infty}^{+\infty}{F\left(kx,ky\right)\cdot e^{j2\pi(kx\cdot x+ky\cdot y)}dkxdky}
\end{equation}

Grâce à la TFi, il est possible de modifier une image dans l’espace de Fourier puis d’obtenir une image résultante, voir même d’obtenir une image à partir d’un espace K seul. La Figure \ref{fig:ft_HF_BF} montre deux cas, le cas ou seulement les basses fréquences ou les hautes fréquences sont conservées. Les basses fréquences expriment le contraste intrinsèque de l’image alors que les hautes fréquences décrivent la netteté et les détails de l’image. C’est pourquoi il est nécessaire de toujours avoir le centre de l’espace K suffisamment défini pour avoir un toutes les nuances de gris de l’image.

\begin{figure}[!htbp]
  \begin{center}
    \includegraphics[width=0.9\textwidth]{Chapitre1/ft_hf_bf.png}
     \end{center}
    \caption{Impact de la sélection des hautes et fréquences dans un espace de Fourier, les basses représentent le contraste de l’image là où les hautes représentent les détails de l’image}
  \label{fig:ft_HF_BF}
\end{figure}

	\subsubsection{Obtenir une image à partir d’un signal RMN et de l’espace de Fourier}

Nous avons vu dans la partie i les différentes étapes amenant à obtenir un signal FID correspondant aux caractéristiques locales du tissu et dans la partie ii qu’il était possible d’obtenir une image en 2D à partir d’un espace fréquentielle appelé espace K. L’objectif in fine est d’obtenir une image ou un volume avec un contraste permettant un diagnostic. 
Le signal S de la FID provenant d’un volume V est donnée par l’équation : 
\begin{equation}
\nonumber
S\left(t\right)=\oint_{V}{M_{x,y}\left(r,t\right)\ dr}
\end{equation}
avec r = position de V à l’instant t.

Nous avons vu que le proton précesse à la fréquence de Larmor, elle-même dépendante du champ magnétique appliqué au proton. S’il est possible de faire varier spatialement $B_0$, alors la fréquence de Larmor changera aussi. En faisant varier spatialement $B_0$, il est possible de remplir un espace de Fourier puis au moyen de TFi d’obtenir une image en 2D. Comme $B_0$ n’est orienté que selon l’axe z, il est seulement possible de le faire varier selon cet axe. Les variations spatiales de $B_0$ sont faites en utilisant des gradients de champ magnétique appelé $G_x$, $G_y$ et $G_z$ pour les axes x,y et z respectivement (Figure \ref{fig:b_gx}). Ils sont exprimés en mT/m. Le nouveau champ B s’écrit : 
\begin{equation}
\nonumber
B=B_0+(0,0,\ x \cdot G_x,+y \cdot G_y+z \cdot G_z)
\end{equation}

\begin{figure}[!htbp]
  \begin{center}
    \includegraphics[width=0.9\textwidth]{Chapitre1/b0_avec_gradient.png}
     \end{center}
    \caption{Application d'un gradient $G_x$ selon l'axe x au champ $B_0$ (orienté selon l'axe z).}
  \label{fig:b_gx}
\end{figure}

La sélectivité spatiale à l’aide de la fréquence de Larmor est faite en deux étapes, l’excitation sélective qui choisira la coupe (par convention l’axe z) à l’aide de $G_z$ (gradient de sélection de coupe) et l’encodage spatiale en 2D avec $G_x$ et $G_y$. 
Pour une position z0 dans l’imageur, en n’utilisant que $G_z$, la fréquence de Larmor à cette position sera : 
\begin{equation}
\nonumber
\nu\left(z0\right)=\ \gamma H\cdot\left(B0+z_0G_z\right)
\end{equation}
L’application de $G_z$, avec un impulsion RF permet de générer une aimantation transversale $M_{x,y}$ sur un volume d’intérêt dont l’épaisseur $\Delta z$ dépendra de la bande passante de l’impulsion RF et de l’intensité de $G_z$ : 
\begin{equation}
\nonumber
\Delta z=\frac{\Delta f}{\gamma H \cdot G_z}
\end{equation}

Le codage spatial 2D est l’étape qui consiste à remplir l’espace K à la position z0.  Par convention, les axes l’espace K sont appelés direction d’encodage de lecture et direction d’encodage de phase. Le gradient d’encodage de phase choisira la ligne de l’espace K qui sera remplie et le gradient de lecture remplira cette ligne.

L’ensemble de ces étapes nécessaire à l’obtention d’un espace de Fourier rempli peut-être schématiser temporellement à l’aide d’un chronogramme (Figure \ref{fig:chrono}). La taille et la résolution de l’espace K est dépendant de la taille de l’image résultante que l’on souhaite obtenir.  Si le champ de vue (FOV, field of view) de l’image souhaitée est de $Fov_x$ et $Fov_y$,  les tailles (en unité spatiale [mm])  de champs de vue selon l’axe x et l’axe y, respectivement, avec une matrice de taille MxN, alors résolution spatiale de l’image sera le rapport entre le champ de vue et la taille de la matrice de l’image( $\frac{{Fov}_x}{M},\frac{{Fov}_y}{N} )$. L’incrément des fréquences spatiales ($\Delta kx$ et $\Delta ky$ ) est donné par les relations $\Delta kx= \frac{1}{{Fov}_x}$ et $\Delta ky= \frac{1}{{Fov}_y}$. 

\begin{figure}[!htbp]
  \begin{center}
    \includegraphics[width=0.9\textwidth]{Chapitre1/chrono.png}
     \end{center}
    \caption{Exemple de chronogramme de Séquence IRM (dites d'écho de gradient) et remplissage de l'espace de Fourier. Etape I : Sélectivité spatiale avec la combinaison d’une impulsion RF et d’un gradient de sélection de coupe. Etape II : Sélection de la ligne de l’espace de Fourier à remplir avec un gradient d’encodage de phase et initialisation de la ligne de lecture en ce plaçant à l’extrémité gauche de la ligne (A vers B) . Etape III : Remplissage de la ligne grâce au gradient de lecture (B vers C puis D). La flèche au niveau du gradient d’encodage de phase symbolise la direction de remplissage de l’espace de Fourier (de bas en haut)}
  \label{fig:chrono}
\end{figure}

Le temps d’un séquence IRM est appelé TR (temps de répétition) et le temps entre l’impulsion RF et la lecture (Figure \ref{fig:chrono}, Etape III) est le temps d’écho (TE). 

La séquence présentée ici est une séquence dite d’écho de gradient, du fait que le signal lu par l’antenne est codé par le gradient de lecture. Le signal varie en T2* (décroissance de la FID) du aux inhomogénéités de $B_0$, l’application d’une impulsion RF de 180° avant le gradient de lecture permet de les corriger et d’obtenir un signal variant en T2 (écho de spin).
Nous avons vu dans la partie i que le temps de relaxation T1 dépend du tissu/milieu, une conséquence de cela est que l’angle de bascule optimum pour avoir le plus d’aimantation à un milieu donné sera défini par l’équation suivante \cite{Ernst1966} : 
\begin{equation}
\nonumber
\cos{\alpha}=\ e^{-\frac{TR}{T1}}
\end{equation}

La combinaison de l’angle de bascule $\alpha$, de TR et du TE permet de remplir l’espace K avec des valeurs différents et donnera \textit{in fine} le contraste de l’image \cite{Markl2012}.

\section{L’Imagerie Pondérée en Diffusion}
\subsection{Principes physiques}
\subsubsection{Le mouvement des molécules d’eau}
Dans la partie précédente, nous avons vu que la FID était dépendante de la fréquence de Larmor, et nous avons utilisé cette propriété pour la faire varier en utilisant un champ magnétique. Elle est aussi dépendante du moment gyroscopique de l’atome d’hydrogène ($^1H$), et comme 90 \% de l’hydrogène du corps humain est localisé dans les molécules d’eau ($H_2O$), le signal RMN est donc dominé par l’eau. La quantité de signal recueilli sera maximal dans une zone avec un grande densité de molécules d’eau et décroira en fonction de cette densité, jusqu’à ne plus avoir de signal s’il n’y a plus de molécules d’eau \cite{Mezer2016}. 

Le mouvement des molécules pendant une examen IRM peut être induit par plusieurs facteurs. Le premier cas est le cas, le plus simple, le sujet se déplace alors la zone à imager se déplace aussi, ce mouvement est communément appelé \textit{bulk} (grossier en anglais), il est caractérisé par un mouvement de plus d’un pixel de toute la zone d’intérêt. 

Un mouvement unidirectionnel des molécules d’eau au sein même d’un voxel sera interprété comme un \textit{flux} de molécules d’eau. L’effet du flux sur le signal RMN se traduit par un déplacement de la fréquence de la FID, c’est pourquoi l’imagerie de phase permet de voir le flux comme par exemple le flux sanguin \cite{Wymer2020}. Un flux trop important (supérieur à un pixel) peut être considéré comme du bulk. 

Le dernier mouvement (celui qui nous intéresse) est le mouvement dit de \textit{diffusion}. Ce mouvement est dû aux interactions des molécules d’eau entre elles. Dans un tissu souple, les molécules d’eau sont proches mais peuvent se déplacer (si les molécules sont toutes liées entre elles, alors l’eau devient de la glace et à contrario, si aucunes molécules ne sont liées alors l’eau devient de la vapeur). Comme elles peuvent se déplacer, elles vont vibrer les unes par rapport aux autres et vont venir stocker de la chaleur sous la forme d’agitation moléculaire. Ce phénomène fut observé par Brown au 19ème siècle, c’est pour on parle de mouvement brownien pour décrire ce phénomène. Le coefficient de diffusion (D) peut être approximée comme la vitesse d’une particule dans un milieu. 

\begin{figure}[!htbp]
  \begin{center}
    \includegraphics[width=0.9\textwidth]{Chapitre1/mouv_h2o.png}
     \end{center}
    \caption{Différence entre le bulk (A), le flux (B) et le mouvement de diffusion (C). Chaque carré rouge représente un pixel/voxel. Le bulk (A) se traduit par une variation d’un groupe de molécules d’eau de plus d’un pixel/voxel, il est causé par le mouvement des organes et/ou de l’échantillon. Le flux (B) est le mouvement d’un groupe de molécules d’eau dans un pixel/voxel unidirectionnel. La diffusion (C) est aussi le mouvement d’un groupe de molécules dans un pixel/voxel mais chaque molécule suit une trajectoire indépendante les unes des autres résultant un mouvement total du groupe dans toutes les directions. Adaptée de Tournier et al \cite{2014}}
  \label{fig:mouv_h2o}
\end{figure}

\subsubsection{Imager la diffusion avec la séquence écho de spin}

Maintenant que le phénomène de diffusion est expliqué, nous allons voir comment il est possible l’imager. Nous avons également appris que la seule information que nous utilisons habituellement pour l'IRM est l'intensité du signal/pixel. L'intensité est dominée par la concentration d'eau (densité de protons). Elle est également influencée par le temps de relaxation du signal tel que T1 et T2. Ici, notre tâche consiste à sensibiliser l'intensité à la quantité de diffusion de l'eau ou à la constante de diffusion. 

Les protons processent à la fréquence de Larmor, dépendante de la valeur du champ magnétique local (B). Sur la Figure \ref{fig:grad_1}, trois protons (donc des molécules d’eau par abus de langage) sont schématisés par des ronds rouge, bleu et vert. Après excitation (permettant d’induire la magnétisation $M_{x,y}$ nécessaire à la mesure du signal), les molécules d’eau processent toutes la même fréquence, résultant trois signaux en phase et donc la somme de ces trois signaux (le signal de Raisonnance Magnétique) est maximale. 
L’application d’un premier gradient va modifier la valeur locale du champ magnétique faisant donc varier localement les fréquences de procession de chaque proton. La somme des signaux va décroitre jusqu’au à ce que les trois signaux soient complétement déphasé. Ce gradient est appelé gradient de déphasage.
Après l’arrêt du gradient de déphasage, les trois protons vont re-processé à leur fréquence initiale (fréquence de Larmor à la valeur de $B_0$) mais chaque signal reste déphasé par rapport aux deux autres signaux. La somme des signaux est nulle dû aux déphasages. 
Un second gradient opposé (en phase) au gradient de déphasage est appliqué. Il est appelé gradient de rephasage car il modifie les phases de chaque signal pour remettre les remettre en phases. La somme des signaux croit jusqu’à ce que le gradient ne soit plus appliqué. Comme le gradient de rephasage est l’opposé du gradient de déphasage, les phases de chaque signal retourne à la phase où elle était avant l’application du gradient de déphasage.
Dans cet exemple, la positon des protons vert,bleu et rouge n'a pas varié, résultant que la somme des signaux apres le déphasage-rephasage est identique (maximale) à la valeur de la somme avant le déphasage-rephasage.
\begin{figure}[!htbp]
  \begin{center}
    \includegraphics[width=0.9\textwidth]{Chapitre1/grad_deph.png}
     \end{center}
    \caption{Exemple de déphasage re-phasage par application de gradients \cite{2014}}
  \label{fig:grad_1}
\end{figure}
\clearpage
Dans le mouvement type flux (mouvement unidirectionnel des protons), les protons déphasés dans un voxel vont se « décaler » (Figure \ref{fig:grad_2}.I), la somme des signaux des protons va donc être déphasée par rapport à la somme sans flux de la Figure \ref{fig:grad_1}.

Maintenant, s’il n’y a que la diffusion qui intervient pendant l’acquisition (Figure \ref{fig:grad_2}.II), les protons vont se déplacer dans le voxel mais rester à l’intérieur du voxel (en majorité, les limites du voxel ne sont pas des barrières physiques). Comme tous les protons déphasés par la séquence de déphasage-rephasage sont toujours dans le voxel, la phase du signal RM est identique à une phase sans diffusion mais l’intensité du signal va quand a elle diminue du fait que les phases (locales) des protons sont différentes les unes par rapport aux autres. Une chute de signal dans un voxel indique une diffusion plus importante dans celui-ci, un couple de gradient bipolaire permet de mesurer la diffusion.

\begin{figure}[!htbp]
  \begin{center}
    \includegraphics[width=0.9\textwidth]{Chapitre1/grad-2.png}
     \end{center}
    \caption{Influence du flux et de la diffusion sur le signal MR. I. Flux présent dans le voxel (les protons bougent dans une seule direction), le signal MR est déphasé par rapport à un voxel sans flux. II. Diffusion présente dans le voxel, les protons dans les rectangles jaunes se déplacent de manière microscopique. Ses déplacements microscopiques entraînent une chute de l’intensité (amplitude) du signal MR \cite{2014}}
  \label{fig:grad_2}
\end{figure}
\clearpage

Le problème de cette approche/séquence (Figure \ref{fig:grad_3}.A : RF – gradient de déphasage – gradient de rephasage) est qu’une partie importante du signal est perdu à cause de la décroissance T2* car le signal est lu à la fin de la séquence (gradient de lecture et de phase après le gradient de rephasage).

Pour pallier à ce problème, il est utilisé une séquence dite d’écho de spin (Figure \ref{fig:grad_3}.B) avec une implusion RF de 90° puis le gradient de déphasage et ensuite une autre impulsion RF de 180° qui va venir déphasé les spins des protons et pour finir le gradient de rephasage mais qui n’est plus opposé au gradient de déphasage (car les protons ont déjà été déphasé par la deuxième impulsion RF). 

\begin{figure}[!htbp]
  \begin{center}
    \includegraphics[width=0.9\textwidth]{Chapitre1/spin_echo_chrono.jpg}
     \end{center}
    \caption{Chronogramme simplifié de deux séquences permettant d’obtenir un signal MR pondéré en diffusion. A : Séquence sans impulsion de refocalisation et gradients bipolaires. B : séquence écho de spin avec une impulsion RF de refocalisation (180°) et deux gradient uni-polaires}
  \label{fig:grad_3}
\end{figure}
\clearpage
\subsubsection{Comment les gradients de déphasage et rephasage influent sur le signal MR}


Nous avons vu dans le paragraphe précédent que la combinaison d’impulsions RF et de gradients de déphasage et de rephasage permettait d’obtenir un signal MR sensible à la diffusion. Pour rappel, les molécules d’eau bougent en fonction de plusieurs paramètres physiques internes et externes au milieu, appelé coefficient de diffusion (D). Pour la suite du chapitre, $S_0$ sera le signal MR sans gradients et S le signal avec.

Plus les gradients seront espacés, plus les molécules d’eau auront le temps de se déplacer, entrainant une baisse plus importante de S. L’intervalle entre deux gradients est noté $\Delta$ . Sur un même $\Delta$, si D est grand, alors l’amplitude de S sera plus faible. 

Le déphasage induit par le gradient de déphasage dépend de la valeur locale du champ magnétique modifiée par la quantité de magnétisation appliqué pendant une durée. Cette modification est le produit de l’amplitude du gradient (G) ainsi que la durée du gradient ( $\delta$ ). Un produit G x $\delta$ élevé entrainera un déphasage élevé. Le déphasage influant sur S, alors G et $\delta$ aussi.

S(t) est donc une fonction de G,D, $\delta$,$\Delta$. 

\begin{figure}[!htbp]
  \begin{center}
    \includegraphics[width=0.9\textwidth]{Chapitre1/param_b.png}
     \end{center}
    \caption{Schéma simplifié d’une séquence écho de spin et les différents paramètres des gradients. G : amplitude du gradient de déphasage/rephasage. $\delta$ : temps d’application du gradient de déphasage/rephasage. $\Delta$ : intervalle temporelle entre les gradients de déphasage et rephasage}
  \label{fig:param_b}
\end{figure}

S peut être décrite par une mono-exponentielle avec G,D$\delta$,$\Delta$ comme paramètres \cite{2014_2}. 
\begin{equation}
\nonumber
S= S_0 \cdot e^{-\gamma^2 \cdot G^2 \cdot\delta ^2 \cdot (\Delta-\frac{\delta}{3}) \cdot D}
\end{equation}

Par convention, on appelle valeur b le produit $\gamma^2G^2\delta^2 (\Delta -\frac{\delta}{3})$. Plus b est élevé alors $e^{-bD}$ tend vers 0 entrainant une baisse d’intensité (globale) S mais un plus grand signal pondéré en diffusion. Pour faire varier b, il est possible de faire varier G,$\delta$,$\Delta$. $\delta$ et $\Delta$ sont des temps donc les accroitre augmentera b mais aussi le temps total d’acquisition.  Accroitre G est la solution envisager dans la plupart des cas mais de fort gradient peuvent endommager l’instrumentation.

\subsubsection{Le coefficient apparent de diffusion (ADC)}

Si nous repartons de la définition de S, le rapport $\frac{S}{S_0}$ est égale à $e^{-\gamma ^2 G^2 \delta ^2 \Delta ^2-\frac{\delta}{3})D}$. Avec le logarithme népérien, nous obtenons la relation suivante :
\begin{equation}
\nonumber
\ln{\left(\frac{S}{S_0}\right)}=\ -\gamma^2 \cdot G^2 \cdot\delta ^2 \cdot (\Delta-\frac{\delta}{3}) \cdot D
\end{equation}
Que nous pouvons transformer pour obtenir une équation affine (y = p – m.x) : 
\begin{equation}
\nonumber
\ln{\left(S\right)}=\ln{\left(\frac{S}{S_0}\right)}=- \gamma^2 \cdot G^2 \cdot\delta ^2 \cdot (\Delta-\frac{\delta}{3}) \cdot D
\rightarrow\ \ln{\left(S\left(b\right)\right)=}\ln{\left(S_0\right)}-\ b.\ D\ 
\end{equation}

Avec y le logarithme népérien de S à une valeur de b fixé [a.u] , p le logarithme népérien de S [a.u]  avec b = 0 [$s/mm^2$] (un signal sans gradient de diffusion/dephasage-rephasage) et  m  = coefficent de diffusion D [$mm^2/s$] et x = valeur b [$s/mm^2$].

\begin{figure}[!htbp]
  \begin{center}
    \includegraphics[width=0.9\textwidth]{Chapitre1/adc.png}
     \end{center}
    \caption{Relation entre l'intensité du gradient et l'intensité du signal. La pondération de la diffusion augmente avec l'intensité du gradient. Cela entraîne une perte de signal, et la pente, mais pas l'intensité absolue du signal, contient des informations sur la constante de diffusion. Une fois la constante de diffusion déterminée à partir de la pente de chaque pixel, il est possible d'obtenir une carte des constantes de diffusion, dans laquelle la luminosité est proportionnelle aux constantes de diffusion ; plus le pixel est lumineux, plus la constante de diffusion est élevée. Dans cet exemple, le pixel rose présente une perte de signal importante, ce qui suggère une diffusion rapide. Sur la carte des constantes de diffusion, cette région devient brillante. Le pixel bleu, en revanche, présente une perte de signal beaucoup plus faible et une pente peu prononcée. La constante de diffusion dans cette région est lente et devient plus sombre dans la carte des constantes de diffusion}
  \label{fig:adc}
\end{figure}
Le coefficient de diffusion D est la propriété physique du tissu, il est indépendant de l’acquisition RMN. Pour le retrouver à partir d’une acquisition de diffusion, il est possible de faire plusieurs acquisitions avec différents b (Figure \ref{fig:adc}). Le coefficient de diffusion trouvé à l’aide la variation de b est appelé coefficient apparent de diffusion (ADC, \textit{apparent diffusion coefficient}). Dans les faits, il est possible de ne faire que deux acquisitions avec deux valeurs b différentes car la variation de diffusion est linéaire en fonction de b.
\clearpage
 \subsection{Décrire la directionnalité d’un milieu à l’aide de la diffusion}

\subsubsection{Le milieu influe sur la diffusion des molécules d’eau}

Dans la parie précédente nous avons vu que les molécules d’eau avaient un déplacement aléatoire (browniens) dans toutes les directions : la diffusion. Cette affirmation est inexacte car la direction de ce mouvement n’est en effet pas dirigée par les molécules d’eau elles-mêmes mais par le milieu qui va contraindre leurs déplacements. 

Une expérience connue consiste à déposer une goutte d’encre sur un papier et d’attendre qu’une tâche se forme. Si le maillage du papier est carré alors, la tâche sera ronde car l’encre diffuse de façon isotrope. Par contre si le maillage n’est plus carré mais rectangulaire dans la direction gauche-droite, alors la tâche sera anisotrope en forme d’ellipse avec comme direction principale(flèche bleue) de cette ellipse l’axe gauche-droite (Figure \ref{fig:grille_encre}). 

\begin{figure}[!htbp]
  \begin{center}
    \includegraphics[width=0.9\textwidth]{Chapitre1/grille_encre.png}
     \end{center}
    \caption{ Schéma représentant la diffusion d’une goutte d’encre sur un maillage isotrope (haut) et sur un maillage anisotrope (bas). Dans le cas du maillage anisotrope (ici plus de fibres ayant une direction gauche-droite que haut-bas).L’encre diffuse dans la direction ayant le plus de fibres (droite-gauche).}
  \label{fig:grille_encre}
\end{figure}
Dans le cas du flux, comme les molécules d’eau ont une orientation unidirectionnelle, leur mouvement peut être modélisé par un vecteur.  
La diffusion isotrope a une forme circulaire, il n’y a donc pas d’orientation préférentielle de la diffusion. La norme d’un vecteur (le rayon du cercle) suffit à décrire la diffusion.
La diffusion anisotrope est décrite par une ellipse, il faut donc deux longueurs, le grand axe (bleue) et petit axe (rouge) au minimum décrire la géométrie de l’ovale et un vecteur pour décrire l’orientation. 

\begin{figure}[!htbp]
  \begin{center}
    \includegraphics[width=0.9\textwidth]{Chapitre1/ellipse.png}
     \end{center}
    \caption{Paramètres nécessaires pour définir une ellipsoïde}
  \label{fig:ellipse}
\end{figure}

En trois dimensions, une ellipsoïde (ellipse en trois dimensions) nécessite six paramètres pour être définie. T trois longueurs sont nécessaires pour le grand, moyen et petit axe, perpendiculaires les uns par rapport aux autres. Ces longueurs sont appelées « valeurs propres » $\lambda_1,\lambda_2,\lambda_3$. Trois « vecteurs propres » $e_1$,$e_2$,$e_3$ sont nécessaires pour définir l’orientation. Parce qu’il faut six paramètres pour définir une ellipsoïde, six mesures de longueurs le long de six axes arbitraires sont nécessaires pour la déterminer.

\subsubsection{L’imagerie pondérée en diffusion pour obtenir la direction d’un milieu}

Dans la partie sur l’encodage de l’espace de fourier, nous avons vu que les gradients de lecture et d’encodage de phase avaient chacun une direction privilégiée (dans le cas d’une acquisition cartésienne : axe z pour la coupe, y pour l’encodage de phase et x pour la lecture). Les gradients de diffusion ont aussi une orientation privilégiée. En acquérant une première image avec un gradient dans une direction puis une deuxième image avec une autre orientation, il est possible de savoir si la diffusion est contrainte dans une orientation spécifique (Figure \ref{fig:grad_4}).

\begin{figure}[!htbp]
  \begin{center}
    \includegraphics[width=0.9\textwidth]{Chapitre1/grad_4.png}
     \end{center}
    \caption{Relation entre l’orientation du gradient, le mouvement des molécules et la perte du signal (S). Quand un gradient avec une orientation horizontale est appliqué, le mouvement des molécules d’eau le long de l’axe horizontale entraine une perte de signal. }
  \label{fig:grad_4}
\end{figure}

En utilisant six images pondérés en diffusion ayant chacune l’orientation des gradients de diffusion différente, il est possible d’obtenir pour chaque pixel/voxel une ellipsoïde décrivant l’anisotropie d’un milieu.
\begin{figure}[!htbp]
  \begin{center}
    \includegraphics[width=0.9\textwidth]{Chapitre1/mileu_on_diff.png}
     \end{center}
    \caption{Influence de mulieu sur la géométrie du tenseur de diffusion. Un milieu anisotrope permettra un tenseur avec une direction principale supérieur aux deux autres directions alors qu'avec un milieu isotrope les trois valeurs propres sont égales.}
  \label{fig:diff_avec_milieu}
\end{figure}
La diffusion anisotropique est d’un grand intérêt en bio-imagerie pour définir et comprendre l’architecture des tissus vivants. Comme certaines parties du corps sont des structures ordonnées, la diffusion des molécules d’eau va être plus intense ($\lambda_1\gg\lambda_2\ \approx\lambda_3$) le long de ces structures (Figure \ref{fig:diff_avec_milieu}). Dans d’autres cas, la diffusion sera isotrope, l’ellipsoïde tend vers la forme d’une sphère traduisant un milieu désordonné ($\lambda_1$=$\lambda_2$=$\lambda_3$).

\clearpage
\subsubsection{Obtenir les paramètres de l’ellipsoïde avec le tenseur de diffusion}

Les six paramètres de l’ellipsoïde (longueurs et vecteurs propres) ne sont pas directement disponibles à partir des six images pondérées en diffusion. Pour cela, l’on utilise un objet mathématique appelé tenseur de diffusion (DT) qui est une matrice 3x3 \cite{Basser1994}: 
\begin{equation}
\nonumber
DT=\ \left(\begin{matrix}D_{xx}&D_{xy}&D_{xz}\\D_{yx}&D_{yy}&D_{yz}\\D_{zx}&D_{zy}&D_{zz}\\\end{matrix}\right)
\end{equation}


Avec $D_{ij}$ coefficient de diffusion dans la direction i,j. DT est une matrice symétrique donc $D_{ij}=\ D_{ji}$. Pour rappel, $\ln{\left(S_{ij}\left(b\right)\right)=}\ln{\left(S_0\right)}-\ b_{ij}$.\ $D_{ij}$, connaissant la valeur de b, la direction du gradient de diffusion, $S_0$ et $S_{j}(b)$, nous obtenons $D_{ij}$.


Pour remplir le DT, il est nécessaire d’avoir les six coefficients de diffusion  $D_{xx}$, $D_{yy}$, $D_{zz}$, $D_{xy}$, $D_{yz}$, $D_{xz}$ qui s’obtiennent en résolvant un système à 7 inconnues ($S_0$ + 6 $S_{ij}$) : 
\begin{equation}
\nonumber
\begin{bmatrix}
ln(S_{xx}) \\
ln(S_{yy})\\
ln(S_{zz})\\
ln(S_{xy})\\
ln(S_{yz})\\
ln(S_{xz}) \\
ln(S_0)
\end{bmatrix}  
= ln(S_0)-
\begin{bmatrix}
{\bar{b}}_{xx} \cdot D_{xx} \\
{\bar{b}}_{yy} \cdot D_{yy}\\
{\bar{b}}_{zz} \cdot D_{zz}\\
{\bar{b}}_{xy} \cdot D_{xy}\\
{\bar{b}}_{yz} \cdot D_{yz}\\
{\bar{b}}_{xz} \cdot D_{xz} \\
0
\end{bmatrix} 
\end{equation}
${\bar{b}}_{ij}$ correspondant à l’orientation  du gradient de diffusion  à la valeur b.

Diagonaliser DT permet d’obtenir les valeurs propres du DT $\lambda_1$,$\lambda_2$,$\lambda_3$ dans les axes du repère de diagonalisation (les vecteurs propres) $e_1$,$e_2$,$e_3$ avec comme direction principale de l’ellipsoïde donnée par $e_1$ et $\lambda_1$. 

\subsubsection{Métriques dérivées du tenseur de diffusion}

Les vecteurs et valeurs propres offrent une information importante sur l’anisotropie du tissu, mais ils ne peuvent être utiliser indépendamment les uns des autres. L’une des méthodes les plus simples est de faire la moyenne géométrique des images $S_{xx}$,$S_{yy}$,$S_{zz}$ : 
\begin{equation}
\nonumber
S_{total}=\ \sqrt{\prod_{i=j=1}^{3}S_{i,j}}=S_0\cdot e^{-\frac{1}{3}\sum_{i=j=1}^{3}{D_{i,j}\ \cdot b}}\ =\ S_0\cdot e^{-\frac{D_{Trace}}{3}\ \cdot b}\ =\ S_0\cdot e^{-ADC\cdot b}\ 
\end{equation}

Le tiers de la trace du DT est égale au coefficient apparent de diffusion ADC. Cette méthode permet d’obtenir l’ADC à partir d’imagerie pondérée en diffusion avec un gradient directionnel défini. 

La Fraction d’Anisotropie (FA) permet d’obtenir un scalaire variant entre 0 et 1 décrivant l’anisotropie du tenseur. Une valeur proche de zéro indique un tenseur isotrope alors qu’une valeur de 1 indique que $\lambda_1\gg\lambda_2\ \approx\lambda_3$.
\begin{equation}
\nonumber
FA=\ \sqrt{\frac{3}{2}\cdot\frac{{(\lambda_1-\ \hat{\lambda})}^2+\ {(\lambda_2-\ \hat{\lambda})}^2+\ {(\lambda_3-\ \hat{\lambda})}^2}{\lambda_1^2+\lambda_2^2+\ \lambda_3^2\ }}
\end{equation}

Avec $\hat{\lambda}$ valeur moyenne des trois valeurs propres $\lambda_1$,$\lambda_2$,$\lambda_3$. 

\subsubsection{Tractrographie à partir d’un tenseur de diffusion}

Jusqu’à présent, nous avons vu l’orientation du tissu dans un seul voxel. Cependant il difficile d’évaluer la direction en 3D du tissu. Pour cela des algorithmes (dit tractographie) permettant d’interpoler les valeurs et vecteurs propres du tenseur pour obtenir des fibres virtuelles (\textit{streamlines}) reliant plusieurs zones du tissu et ainsi obtenir une trajectoire globale (plusieurs voxels) des fibres dans le tissu. 

Une fibre a besoin d’un point de départ (appelé graine ou \textit{seed} en anglais), d’une distance minimale et maximale, et d’une valeur discriminante les voxels voisins. Une tractographie est dites \textit{déterministe} si chaque graine donne une seule fibre. A contrario, elle sera \textit{probabiliste} si plusieurs fibres sont produites à partir d’une même graine.

Il existe plusieurs classes d’algorithmes permettant d’obtenir ces fibres, la classe présentée ici est appelé « Affectation des fibres par suivi continu (FACT) » \cite{Mori1999}. L’idée principale de cette approche est de suivre l’orientation locale par petit pas/incrément à partir d’une graine. Si le pas est trop important alors il en suivra une estimation biaisée de la trajectoire de la fibre (Figure \ref{fig:FACT}).

\begin{figure}[!h]
  \begin{center}
    \includegraphics[width=0.9\textwidth]{Chapitre1/fact.png}
     \end{center}
    \caption{Exemple de tractrographie FACT. Les doubles flèches représentes l’orientation locale du tissu dans chaque voxel. Les points noirs représentent un tractographie avec un incrément suffisamment petit pour obtenir une bonne approximation de la trajectoire globale des fibres du tissu. Les points blancs sont trop espacés les uns des autres pour obtenir une trajectoire représentative}
  \label{fig:FACT}
\end{figure}
\clearpage
En plus du pas, pour éviter les mauvaises trajectoires, un seuil de FA est utilisé pour éviter les trajectoires dans les voxels étant trop isotropes. Ce seuil est couplé à un critère/seuil basé sur la courbure de la streamline, une fibre musculaire ne pouvant faire un angle droit. La courbe de la streamline doit être inférieure à ce seuil pour être ajouté à la streamline. Grâce à ces deux critères, il est possible d’éliminer les fausses approximations (Figure \ref{fig:impact_FA}). 

 \begin{figure}[!h]
  \begin{center}
    \includegraphics[width=0.9\textwidth]{Chapitre1/impact_fa_tracto.png}
     \end{center}
    \caption{Trois tractographies avec une graine identique, sans seuil de FA et de courbure (gauche), avec seulement le seuil de FA (milieu) et avec le seuil de FA et le seuil de courbure (droite). Code couleur : vert : antérieur-postérieur. Rouge : gauche-droite. Bleu : supérieur-inférieur (tête -pied).
Les fibres peuvent être masqué à l’aide de volume binaire ainsi que certaines graines peuvent être exclus du calcul de la tractrographie}
  \label{fig:impact_FA}
\end{figure}

Les fibres peuvent être masquées à l'aide d'un volume binaire. Des graines peuvent aussi être exclues du calcul de la tractographie.
\clearpage
\subsubsection{Segmentation des voxels avec une orientation préférentielle à l’aide de la tractrographie}


La tractographie permet d’obtenir une vision globale de l’orientation des fibres dans un volume 3D. Il est possible de filtrer la tractographie avec une orientation définie, cela consiste à ne garder que les fibres ayant cette orientation. Ensuite, le nombre total de fibres dans un voxel d’une grille cartésienne est compté, résultant un volume scalaire avec le nombre de fibre généré par la tractographie dans chaque voxel (Figure \ref{fig:TDI}). Cette technique est appelé \textit{Track-Density Imaging} (TDI) \cite{Calamante2010}, elle fut créée dans le but d’obtenir des images anatomiques de la matière blanche à partir d’imagerie de diffusion directionnelle. La grille cartésienne a une taille arbitraire, elle n’est pas nécessairement de la même taille et même résolution que les volumes pondérés en diffusion utilisés pour la tractographie. 

\begin{figure}[!h]
  \begin{center}
    \includegraphics[width=0.9\textwidth]{Chapitre1/TDI.png}
     \end{center}
    \caption{ Carte en densité de fibres obtenue à partir d’une tractographie déterministe et d’un filtre directionnel appliqué sur cette tractographie. En ne gardant que les voxels où la densité de fibre est supérieure à zéro, nous obtenons un masque binaire des fibres ayant l’orientation inférieur-supérieur}
  \label{fig:TDI}
\end{figure}
\clearpage



\subsubsection{Les limites du tenseur}
L’IRM de diffusion permet d’obtenir la diffusion des molécules d’eau dans une direction privilégiée et le modèle du DT traduit cette information en une ellipsoïde à une orientation privilégiée e1. Un problème se pose dans certaines zones du corps humain où il existe de manière locale plusieurs populations de fibres ayant des orientations différentes (Figure \ref{fig:dti_limite}), c’est notamment le cas dans la matière blanche. Le tenseur est aussi dépendant d’un signal décrivant une mono-exponentielle pour être calculé. Cela n’est plus le cas comme par exemple dans un modèle perfusé décrit ci-dessus. Ces limites peuvent induire des chutes de FA indiquant une mauvaise représentation par le DT de la diffusion des molécules d’eau.

 \begin{figure}[!h]
  \begin{center}
    \includegraphics[width=0.9\textwidth]{Chapitre1/probleme_dt.png}
     \end{center}
    \caption{Différentes configurations de fibres et leurs tenseurs de diffusion correspondant. Les populations de fibres avec une seule orientation permettent d’obtenir une ellipsoïde allongée mais dans le cas où il existe plusieurs populations, l’ellipsoïde devient une sphère}
  \label{fig:dti_limite}
\end{figure}

	\subsubsection{Le modèle HARDI (High Angular Resolution Diffusion Imaging)}
De nombreuses approches ont été développé à partir des années 2000 pour résoudre ce problème de tenseur mal exprimé \cite{Tuch2004} \cite{Descoteaux2009} \cite{Tournier2007}. Ces différentes méthodes permettent d’obtenir une objet mathématique appelé fonction de densité d’orientation de fibres (dODF) qui traduit comme son nom l’indique, les différentes orientations de fibres dans un voxel. L’approche qui nous intéresse est appelée modèle mixte permet d’obtenir directement l’orientation des fibres. Elle se base sur l’hypothèse que les molécules d’eau se déplacent plus lentement que le temps d’acquisition d’une image. L’algorithme le plus simple de modèle mixte est l’estimation d’un multi-tenseur, chaque orientation de fibre est modélisée selon un DT : 
\begin{equation}
\nonumber
S\left(\underline{\hat{u}},b\right)=\ S_0\cdot\sum_{i=1}^{N}{f_i\bullet e}^{-b{\underline{\hat{u}}}^TD_i\underline{\hat{u}}}\ 
\end{equation}


Avec $\underline{\hat{u}}$ orientation du gradient de diffusion et N le nombre d’orientation de fibre dans le modèle. Chaque orientation (ou compartiment) est caractérisée par un coefficient $f_i$ et la diffusion intrinsèque $D_i$. Avec cette nouvelle définition, un nombre N d’orientation de fibres donnant une dODF discrète (N éléments) est disponible mais elle est instable \cite{2014_3}. 

Cette approche en multi compartiment (multi-tenseur) peut être généralisé pour définir la dODF non comme une fonction discrète mais une fonction continue. Le signal de diffusion S n’est la somme de chaque compartiment mais comme étant la convolution sphérique entre la dODF et le un signal R correspondant à un signal pondéré en diffusion d’un voxel avec une seule fibre (R) (Figure \ref{fig:conv_sh}). 

 \begin{figure}[!h]
  \begin{center}
    \includegraphics[width=0.9\textwidth]{Chapitre1/conv_sh.png}
     \end{center}
    \caption{L'ODF F est calculé à partir de la déconvulation entre un signal DWI S et une estimation d'un voxel possédant une seule orientation de fibre R. Ici S est la combinaison de deux orientations de fibres 1 et 2}
  \label{fig:conv_sh}
\end{figure}

La dODF est obtenue en faisant la déconvolution sphérique entre le signal S et R \cite{Tournier2007}. Pour rappel, plus la valeur de b est élevée, meilleur sera l’estimation de la directionnalité de la diffusion. Dans le cas de la neurologie une valeur de b supérieure à 3000 $s/mm^2$ est utilisée. 

 \begin{figure}[!h]
  \begin{center}
    \includegraphics[width=0.9\textwidth]{Chapitre1/fiber_config.png}
     \end{center}
    \caption{Différentes configurations de fibres dans un voxel et leurs impacts sur le DT et la dfODF.  Adaptée de \cite{Sotiropoulos2017}}
  \label{fig:fiber_config}
\end{figure}


La déconvolution sphérique est calculée avec à l’aide des harmoniques sphériques $Y_l^m$, des fonctionnant définissant la surface d’une sphère. Les images dODF contiennent les coefficients de l’harmonique sphérique. Elles peuvent être considérées comme l’équivalent des séries de Fourier pour les fonctions en coordonnées sphériques (r,$\Theta$,$\phi$).  
\begin{equation}
\nonumber
Y_l^m\left(\theta,\phi\right)=\ \sqrt{\frac{\left(2l+1\right)\left(l-m\right)!}{4\ \pi\ \left(l+m\right)!\ }}\ P_l^m\cos{(\theta)}e^{i\omega\phi}\ 
\end{equation}


Avec l dégrée et m ordre de l’harmonique sphérique Y. $P_l^m$ est le polynôme de Legendre correspondant à l et m. 
Une fonction f($\Theta$,$\phi$) est définie comme un développement en harmonique sphérique (d’où l’analogie avec les séries de Fourier) : 
\begin{equation}
\nonumber
f\left(\theta,\phi\right)=\ \sum_{l=0}^{l_{max}}\sum_{m=\ -l}^{l}c_l^mY_l^m\left(\theta,\phi\right)
\end{equation}

Les images contenant les coefficients harmoniques sphériques sont stockées sous la forme d’image à 4 dimensions avec chaque coefficient le long de la quatrième. Les coefficients sont ordonnés dans un ordre défini par différentes conventions fixés par les logiciels de visualisation d’IRM de diffusion.\cite{Tournier2019} Le nombre de volume N de l’image dODF est dépendant de $l_{max}$ (le dégrée maximale d’harmonique sphérique). $N=\frac{1}{2}\ \left(l_{max}+1\right)\left(l_{max}+2\right)$
\\
Exemple : un degrée $l_{max}$ = 8 donne N = 45 volumes.
\\


\section{Exploration structurelle du myocarde }
Dans le paragraphe précèdent, nous avons vu que le cœur avait une organisation structurée (orthotropie) mais que des défauts structurels (remodelage structurel, remplacement des myocytes par du collagène et/ou des cellules adipeuses) entrainent des complications de la fonction cardiaque. 
%
Dans l’optique d’identifier les maladies cardiaques, de les différencier et de les comprendre de nombreux outils ont été développés comme l’électrocardiogramme (ECG) qui permet d’obtenir l’activité du cœur et de connaitre la vitesse et le rythme cardiaque. L’ECG est, avec la prise de tension artérielle (la pression exercée sur les artères pendant la systole et la diastole) le premier examen cardiaque. Ces examens ne permettent pas d’obtenir d’informations sur la localisation des défauts locaux et structurels du myocarde, c’est pourquoi des méthodes d’imagerie sont développées à travers le monde. 

La partie suivante, une description rapide des techniques d’imagerie sera donnée, d’abord les techniques \textit{in vivo}, sur un sujet vivant, puis \textit{ex vivo}, sur un échantillon, les méthodes \textit{ex vivo} étant par nature avec une plus grande résolution mais avec un temps d’acquisition bien supérieure à l’\textit{in vivo} et certaines méthodes sont aussi destructives pour le tissu biologique.

\subsection{L'imagerie \textit{in vivo}}

Identifier une pathologie cardiaque, c’est bien, le faire quand le patient est vivant, c’est mieux. Selon l’OMS (Organisme Mondiale de la Santé), 32\% des décès (17.9 millions de personne)  dans le monde sont dû à des maladies cardiovasculaires. 

Les cardiologues possèdent un vaste arsenal d’imagerie pour identifier les pathologies cardiaques et chacune de ces techniques présentes des avantages et des inconvénients. La suite de ce paragraphe sera un sommaire des techniques utilisées pour étudier la structure du myocarde.



\subsubsection{L’échographie et échographie ultrarapide}

L’échographie cardiaque est un examen cardiaque non invasif, non-ionisant, indolore et il peut être réalisé directement en consultation chez le cardiologue ou dans le lit d’un patient aux urgences. Les ultrasons sont produits par une sonde, vont pénétrer dans la cage thoracique puis être réfléchies par la configuration interne du thorax et produiront un écho en retour. Les échos sont captés par la sonde et une image en deux dimensions sera formé. La résolution spatiale de l’échographie est dépendante de nombreux paramètres comme la durée de l’impulsion ultrasonore ainsi que la profondeur du champ de vue. 

En plus de l’échographie cardiaque, un examen doppler est effectué pour analyser les flux sanguins à travers les cavités et le passage par les valves. 

L’échographie ultrarapide est une technique émergeante d’imagerie échographique prometteuse reposant sur la capacité de contrôler tous les éléments de la sonde échographique. Grâce à cette technique, il est possible de remonter à l’orientation des myofibres et à l’anisotropie des tissus biologiques avec imagerie en tenseur de rétrodiffusion \cite{Papadacci2014} (BTI, \textit{Backscatter Tensor Imaging}) et/ou \textit{Elastic Tensor Imaging} (ETI), une technique qui se base sur le fait que les ondes de cisaillement se propage plus rapidement dans le sens des fibres qu’entre les fibres \cite{Agger2020} . 

\begin{figure}[!htbp]
  \begin{center}
    \includegraphics[width=0.9\textwidth]{Chapitre1/echo_v2.png}
     \end{center}
    \caption{Matériels et imagerie ultrasonore. A : un échographe avec sa sonde. B : image dites en 4 cavités d’un cœur humain avec les deux ventricules et les deux oreillettes. C : orientation des myofibres dans le ventricule gauche ex vivo d’un cochon. L’orientation des myofibre entre l’endocarde et l’épicarde respecte la règle d’orientation hélioïdale des myoyctes dans le VG.   Adaptée de https://www.ccjj.fr/echographie-cardiaque et https://www.fo-rothschild.fr/patient/actes-medicaux/echocardiographie-transthoracique et \cite{Papadacci2014}}
  \label{fig:fig_echographie}
\end{figure}

Pendant de nombreuses années, le traitement de l’imagerie ultrarapide n’était pas mis en œuvre à cause de problème de reconstruction d’images (plusieurs milliers d’images ultrasonore sont nécessaire pour obtenir en volume en trois dimensions du cœur). Avec des ordinateurs de plus en plus rapide et puissant (augmentation de la capacité de la RAM et de la fréquence d’Horloge), les algorithmes de traitement d’imagerie ultrarapide sont réalisables dans un temps de plus en plus court. 
 
\clearpage
\subsubsection{La tomodensitométrie ou tomographie aux rayon X}

La tomodensitométrie (TDM, communément appel scanner) est une technique d’imagerie non invasive qui utilise les propriétés des rayons X pour obtenir un volume en trois dimensions du cœur.  Le scanner est constitué d’un anneau rotatif autour de la table où est allongé le patient. Un tube émetteur de rayon X et des récepteurs/capteurs permettent pendant la rotation de l'anneau, de balayer la zone à étudier. Les rayons X seront plus ou moins absorbées en fonction de la densité du tissu donnant un contraste en niveau de gris dépendant de cette atténuation. La tomographie est utilisée en cardiologie pour étudier le myocarde et les coronaires. Pour avoir un contraste plus élevé dans les vaisseaux sanguins (coronaires par exemple), il est possible d’utiliser des agents de contraste radio-opaque (en blanc sur les images rayon X) à base d’iode (dit agents de contraste iodés). Avec un contraste et en acquérant plusieurs images à la suite, il est possible de d’obtenir le flux sanguin dans les vaisseaux. La résolution spatiale d’une image TDM est de 500 $\mu$ m isotropique pour un temps d’acquisition de l’ordre de la minute.

La TDM est par contre une technologie ionisante, la dose de rayonnement est largement supérieure à la quantité de rayonnement d’une radiographie thoracique. La TDM est aujourd’hui responsable de la majorité de l’exposition au rayonnement. Il est impossible d’avoir un suivi temporel rapproché de l’évolution d’une pathologie avec la TDM.

\subsubsection{L’imagerie par résonnance magnétique }

Les deux méthodes d’imagerie présentées ci-dessus ont chacune leurs avantages (rapidité d’exécution, résolution spatiale, coût) mais aussi des inconvénients (coût en ressources informatiques, dose de rayonnement).  

L’imagerie par résonnance magnétique (IRM) est une technologie d’imagerie basé sur le phénomène de résonnance magnétique nucléaire (RMN) décrit dans le paragraphe suivant du chapitre d’introduction. Ce phénomène physique fut détecté au début des années 1950 et permit depuis le développement de scanner IRM. Une instrumentation adéquate (champ magnétique puissant, antenne d’émission et de réception précises…) et les séquences IRM (chronogrammes de fonctionnement de l’instrumentation changeant de manière locale le champ magnétique) ont favorisé l’émergence de différents contrastes basés sur les temps de relaxation des tissus (T1 et T2, temps nécessaire au spin du proton pour revenir à sa position d’équilibre, voir chapitre Introduction Partie 3).

 L’imagerie médicale par IRM s’appuie sur l’atome d’hydrogène (H) présent en grande quantité dans les tissus mous ou biologiques. Il existe de nombreuses séquences, chacune permettant d’obtenir un contraste dit pondéré T1 ou T2 (la valeur du signal reçu par l’antenne de réception dépend plus du T1 ou du T2).  Les séquences pondérées T1 donneront un contraste permettant de visualiser la graisse et le myocarde, les séquences pondérées T2 seront utiles pour évaluer des œdèmes. En pratique, une substance avec un T1 long et un T2 long (eau) donnera un hyposignal dans une séquence pondérée T1 et un hypersignal avec une séquence pondérée T2. 

Les images résultantes de ces séquences sont dites subjectives, le diagnostic est dépendant de l’interprétation du radiologue, basé sur les changent d’intensité des images pondérées. Pour éviter un diagnostic dépendant d’une tierce personne, des cartes de T1 et T2 sont obtenues à partir de séquences IRM spécifique, les images sont maintenant quantitatives. La valeur de chaque pixel donne une information et des abaques ou dictionnaires de valeurs de T1 et T2 existent dans la littérature (chaque valeur de T1 et T2 existent pour un champ magnétique précis).
\text
L’IRM cardiaque \textit{in vivo} fait face à de nombreux défis. Le cœur est un organe dynamique et son mouvement est provoqué par son propres rythme de contraction et il se déplace aussi dans la cage thoracique avec le mouvement de respiration. C’est pourquoi il fut créé des séquences d’acquisition rapide synchronisées sur le cycle respiratoire et cardiaque pour obtenir un volume en trois dimensions du cœur indépendamment de tout mouvement.Certaines sont même réalisées en apnée (pour ne pas être perturber par la respiration) et ainsi n’obtenir que le mouvement cardiaque (diastole et systole) donnant une image en quatre dimensions (trois dimensions spatiales et une dimension temporelle), les séquences cinétiques. De plus, un examen IRM cardiaque est plus long qu’une échographie ou une TDM cardiaque (30 à 60 min). Comme pour la TDM, il est possible d’utiliser des agents de contraste et dans le cas de l’IRM cardiaque, ils permettent de révéler certaines structures du myocarde comme la fibrose cardiaque (remodelage structurel du myocarde avec une remplacement des myocytes par de la MEC). La résolution de l’IRM in vivo se situe entre 1 et 2 mm isotropie (variable selon les séquences et les conditions d’examens).

\subsubsection{Les contrastes de l’IRM cardiaque \textit{in vivo}}

Un examen IRM cardiaque permet d’obtenir plusieurs informations sur la structure du myocarde. L’examen commence par le placement des plans de coupes puis à des séquences de cinétiques pour obtenir le mouvement cardiaque et la fraction d’éjection (pourcentage de sang éjecté des ventricules). Ensuite une cartographie T1 est réalisée et un agent de contraste à base de gadolinium est administré et une cartographie T1 est faite post injection. Pour finir une séquence en deux ou trois dimensions est réalisée pour obtenir un contraste pondéré T1 ou la fibrose cardiaque ressortira en hypersignal. La recherche en IRM étant en constante évolution, de nouvelles séquences raccourcissant la durée d’acquisition d’une image, d’obtenir une information sur l’architecture du myocarde ou alors de nouveaux contrastes révélant des pathologies jusqu’ici invisibles sans agent de contraste ont vu le jour. 

\subsubsection{Le rehaussement tardif à base de gadolinium }

Chaque tissus/substances possèdent son propre T1 et T2, la repousse longitudinale (chapitre introduction partie 3), est différente d’un tissu à l’autre. L’amplitude du signal de magnétisation, en valeur réelle, suit une loi mathématique exponentielle décroissante qui converge à une valeur maximale correspondant à une repousse longitudinale complète. Plus le T1 est court, plus la repousse converge rapidement vers sa valeur maximale. La repousse passe par zéro (Figure \ref{fig:gado}.A) à un temps dit d’inversion (TI, en valeur absolue le signal est toujours positif mais décroissant avant l’inversion puis croissante), c’est-à-dire qui si l’acquisition d’une séquence est retardée de TI, le signal du tissu associé à ce TI sera nul \cite{Jenista2023}.

\begin{figure}[!htbp]
  \begin{center}
    \includegraphics[width=0.9\textwidth]{Chapitre1/gado_cine_signal.png}
     \end{center}
    \caption{: Influence du T1 et d’un agent de contraste à base de gadolinium sur le signal en IRM.}
  \label{fig:gado}
\end{figure}

Après l’injection d’un agent de contraste à base de gadolinium, celui-ci se distribue dans tout le cœur. Il va alors s’accumuler sur les zones où le myocarde est le plus endommagé comme la fibrose ou les cicatrices (Figure \ref{fig:gado}.B). Le gadolinium étant paramagnétique, il produira un hypersignal (pixel blanc) lors de l’acquisition. 

\begin{figure}[!htbp]
  \begin{center}
    \includegraphics[width=0.9\textwidth]{Chapitre1/gado_paterne.png}
     \end{center}
    \caption{Plusieurs schémas de rehaussement tardif au gadolinium (LGE). les motifs sous-endocardiques (a) et transmuraux (b) indiquent une cardiomyopathie ischémique, la distribution parcellaire (c) et au milieu de la paroi (d) indique une cardiomyopathie non ischémique (telle que la cardiomyopathie dilatée et hypertrophique), un schéma sous-épicardique (e) est typiquement post-inflammatoire (comme la myocardite), un schéma sous-endocardique diffus (f) est typique de l'amyloïdose ou, dans des situations plus rares où le LGE est plus fin et moins diffus, de la fibrose endomyocardique. Traduit et modifié d'après \cite{Barison2022-ok}}
  \label{fig:gado_paterne}
\end{figure}

En choisissant un TI annulant le signal du myocarde, le contraste entre le myocarde sain et la fibrose sera maximal. L’image résultante est appelée imagerie de rehaussement tardif (Late Gadolinium Enhaucement, LGE). En fonction et du paterne de la position du rehaussement  \cite{Barison2022-ok}, il est possible d’identifier la pathologie cardiaque (Figure \ref{fig:gado_paterne}).
\clearpage
\subsubsection{La diffusion cardiaque \textit{in vivo} }
Il est possible d’obtenir l’orientation des fibres musculaires grâce à la RMN en utilisant l’Imagerie Pondérée en Diffusion ou IRM de diffusion (Diffusion Weighted Imaging, DWI). Le chapitre (chapitre introduction partie 3) est consacré à l’explication mathématiques et physiques permettant d’obtenir l’orientation des fibres. 

Le DWI existe depuis les années 1980 \cite{LeBihan1986}, elle fut utilisée chez des patients ayant eu un accident ischémique (arrêt de l’apport sanguin à un tissu/organe).  De nos jours, l’importance de cette technique n’est plus à prouver, elle est devenue une technique standard dans un protocole d’examens neurologique, et à ce jour, le DWI est utilisé en majorité en neurologie, bien qu’elle soit utilisée aussi sur le foie, les reins et pour identifier des tumeurs. C’est grâce de nouvelles méthodes d’acquisition et d’accélération comme l’imagerie echo-planar (\textit{Echo Planar Imaging}, EPI) \cite{Mansfield1977} \cite{PoustchiAmin2001} que le DWI a pu devenir une réalité clinique. L’EPI est une technique qui permet d’acquérir l’ensemble d’une coupe en une seul écho mais qui induit des artéfacts sur l’image résultantes. (Chapitre Introduction partie 3). 

\begin{figure}[!htbp]
  \begin{center}
    \includegraphics[width=0.9\textwidth]{Chapitre1/dti-invivo.jpg}
     \end{center}
    \caption{DTI sur volontaire sain et sur un patient (haut) avec hypertrophie cardiaque (bas) en systole et en diastole. Les metriques de diffusion (E2A, voir chapitre introduction Partie 3) cardiaque sont modifiées avec la pathologie}
  \label{fig:dti_inivo}
\end{figure}

L’IRM de diffusion est un candidat idéal à l’étude de l’architecture du myocarde mais le mouvement rapide et potentiellement irrégulier (le battement cardiaque est certes périodique mais la fréquence varie en fonction de divers paramètres comme d’éventuelles pathologies ou le stress du patient dans le scanner) rendent complexe l’application de l’IRM de diffusion comme examen clinique de routine (résolution de 2x2x8 mm et un temps d’acquisition proche des 30 minutes). 

Néanmoins, le potentiel immense de l’IRM de diffusion cardiaque et le travail de nombreuses équipes à travers le monde ont permis d’obtenir des images pondérées de diffusion \textit{in vivo} du cœur en systole et en diastole (Figure \ref{fig:dti_inivo}) ouvrant la voie à une méthode non-invasive, en trois dimensions et sans agent de contraste pour l’étude des pathologies cardiaques.
\clearpage
\subsubsection{Les plans de coupe d’imagerie cardiaques}

 Pour permettre un diagnostic précis, complet et reproductible, en plus des axes orthogonaux du scanner (axial/transverse), coronale et sagittale), trois coupes sont utilisées en IRM cardiaque : 
 \begin{bulletList}
\item Vue en 2 cavités (2CAV) : visualisation des parois antérieur et postérieur d’un ventricule, ainsi que l’apex. 
 \item Vue en 4 cavités (4CAV) : visualisation des deux ventricules, des deux oreillettes, les valves tricuspide et mitrale et des septums intraventriculaire et intraauriculaire, elle aussi appelée vu en grand axe ou \textit{long axis} (LA).
 \item Vue en petit axe (PA) : visualisation des deux ventricules perpendiculairement à la vue en 4 cavités, elle aussi appelée en \textit{short axis} (SA). 
\end{bulletList}

\begin{figure}[!htbp]
  \begin{center}
    \includegraphics[width=0.9\textwidth]{Chapitre1/coupes_IRM.png}
     \end{center}
    \caption{Plans de coupes usuelles en imagerie et en imagerie cardiaque.Gauche : plans usuels de l'imagerie. Droite : les plans de l'imagerie cardiaque}
  \label{fig:coupes_IRM}
\end{figure}

\subsubsection{Conclusion sur l’imagerie \textit{in vivo}}

De nombreuses modalités d’imagerie permettent aux cardiologues et radiologue d’identifier les pathologies cardiaques de manière non-invasive et indolore. Il est même possible de remonter à l’orientation des chaînes de myocyte grâce à l’échographie ultrarapide et à l’IRM de diffusion. Cependant, le mouvement cardiaque, le temps d’acquisition et le mouvement du patient sont de véritables problèmes pour obtenir une image en haute résolution de l’orientation des cardiomyocytes.

Pour mieux comprendre et décrire l’anatomie cardiaque, il est alors nécessaire de prélever le cœur et de l’étudier avec des techniques d’imagerie ( et potentiellement détruire le tissu) plus fines que celles employés pendant un examen cardiaque. 

\subsection{L’imagerie \textit{ex vivo}}

La partie imagerie \textit{in vivo} a été introduite avec le postulat qu’il est important d’identifier une pathologie cardiaque quand le patient est toujours vivant. Malheureusement, les méthodes \textit{in vivo} ne sont pas capable d’obtenir une image suffisamment résolue pour obtenir une image nette d’une cellule cardiaque ainsi que de son fonctionnement interne.  

Les premiers travaux d’étude de l’architecture cardiaque ont été réalisés sur des cœurs \textit{ex vivo} (de l’Égypte Antique à l’ère moderne en passant par la renaissance avec les dessins du cœurs trouvés dans les carnets de Léonard de Vinci). Cela a permis une meilleure compréhension de fonctionnement du cœur : du fonctionnement interne d’une cardiomyocyte (contraction suite à une excitation électrique) jusqu’à l’architecture laminaire du myocarde en passant par la découverte du réseau (fibres) de Purkinje et des cellules de Purkinje (par Jan Evangelista Purkinje au XIXème siècle).

 Malgré le caractère sans retour de l’imagerie \textit{ex vivo}, c’est grâce à elle que de nombreuses techniques d’imagerie (comme le LGE) ont pu être validées et être utiliser maintenant comme diagnostics des pathologies cardiaques. La suite de cette partie sera consacrée à présenter les différentes techniques d’imagerie \textit{ex vivo} cardiaque.

\subsubsection{L’Histologie}

Le mot histologie vient du grec \textit{histos} (tissus) et \textit{logos} (discours, parole), c’est l’étude des tissus biologiques. La base de l’histologie est la dissection d’un cœur. Pour obtenir une imagerie haute résolution du myocarde au microscope, il est nécessaire de préparer le tissu avant l’imagerie. D’abord le cœur est découpé/divisé en plusieurs région d’intérêt. Le morceau de tissu sera ensuite déshydraté puis coupé dans le sens voulu (pour aider à la découpe, le bloc de tissu est immergé dans de la paraffine), chaque tranche a une épaisseur de quelques $\mu$m. Chaque coupe sera ensuite déposée sur une lame avant la coloration. Enfin la coupe de tissu colorée est scanner à la l’aide d’un microscope avec une résolution numérique de l’ordre du $\mu$m.

La coloration est une étape cruciale, elle permet de mettre en évidence des éléments du tissu (cellules adipeuses, myocytes, cellules de Purkinje …) et de les différencier entre eux. 

Le trichrome de Masson est très populaire car il permet de différencier les fibres de collagènes des fibres musculaires. Sur l’image, les fibres musculaires apparaitront en rouge, les cytoplasmes (substance intra-cellulaire) en rose , en bleu/vert le collagène et en noir les noyaux des cellules. Elle fait intervenir trois colorations successives : coloration des noyaux (hémalun de Mayer ou Hématoxyline de Harris), coloration du cytoplasme (mélange de fuchsine acide et de rouge ponceau) et ensuite une coloration du collagène par le vert lumière (ou bleu aniline)\cite{Golberg2024}. Grâce a ce contraste, il est possible de voir les chaînes de myocytes entourés par la MEC.

\begin{figure}[!htbp]
  \begin{center}
    \includegraphics[width=0.9\textwidth]{Chapitre1/histo.png}
     \end{center}
    \caption{Quatre coupes d’histologie avec les colorations rouge picro-sirius (ligne du haut) et trichrome Masson (ligne du bas) dans deux cas de fibrose, remplacement (gauche) et interstitiel (droite). Les flèches noires pointent vers l’accumulation de collagène dans le tissu cardiaque. La fibrose est en rose/rouge sur le rouge sirius et en bleue sur le trichrome de Masson. Le tissu cardiaque est visible en rose/rouge sur Masson et en jaune \/ beige sur la coloration rouge sirius. Le contraste est plus élevé entre la fibrose et le tissu sur l’image coloré en rouge sirius mais les chaînes de myocyte sont visibles sur l’image colorée avec le trichrome de Masson. Adaptée de \cite{Qi2022}}
  \label{fig:histo}
\end{figure}

Le rouge picro-sirius est une autre coloration utilisée pour visualiser le collagène, et différencier différents type de collagène présents dans les tissus biologique si le microscope utilise une lumière polarisée\cite{Puchtler_1973} . Ici le collagène est rouge (avec de la microscopie classique, la couleur devient verte ou orange en fonction du type de collagène avec de la lumière polarisée) alors que le cytoplasme et les fibres musculaire sont jaunes. 

L’histologie est la méthode d’imagerie \textit{ex vivo} de référence (ou \textit{gold standard}) pour l’étude du tissu cardiaque et de l’orientation des myofbres grâce à sa capacité d’obtenir des images avec une grande résolution, de différencier les composantes du tissu cardiaque malgré le une destruction du cœur et une image en deux dimensions. 

\subsubsection{Fluorescence et clarity imaging}

Le cœur possède une architecture en trois dimensions, donc réduire l’étude de la structure cardiaque à seulement deux n’est pas idéal pour étudier l’orthotropie du myocarde. La microscopie en fluorescence repose sur la formation d’une image par détection de la lumière émise par l’objet. L’objet est illuminé avec une source lumineuse avec une seule longueur d’onde (laser ou lumière blanche filtrée) puis les fluorophores émettent de la lumière, cette lumière sera filtrée pour ne garder que la longueur d’onde souhaité dans l’image résultante. Une image nette est obtenue en microscopie optique si l’image est située dans le foyer image de l’objectif, en déplacement la lentille, le foyer se déplace et il est possible d’obtenir un volume tridimensionnel.

Les tissus biologiques ne sont pas transparents, le processus d’imagerie par microscope est donc limité. 

CLARITY( Clear, Lipid-exchanged, Acrylamide-hybridized, Rigid, Imaging/immunostaining-compatible, Tissue hYdrogel) \cite{Du_2018} est une technique d'imagerie avancée développée pour rendre les tissus biologiques transparents, tout en préservant leur structure moléculaire proposé en 2013. Le processus CLARITY consiste à remplacer les lipides, qui contribuent à l'opacité des tissus, par un hydrogel transparent. Une fois les lipides extraits, le tissu devient optiquement transparent, permettant ainsi l'observation de détails fins à l'intérieur des tissus, tels que les connexions neuronales dans le cerveau, à l'aide de techniques d'imagerie en profondeur, comme la microscopie à fluorescence.

\begin{figure}[!htbp]
  \begin{center}
    \includegraphics[width=0.9\textwidth]{Chapitre1/figure_clarity_v2.png}
     \end{center}
    \caption{Clarity imaging sur cœur de rat ex vivo. A : protocole de préparartion du cœur. B : cœur avant et après avoir été éclairci. C : volume en trois dimensions d’un réseau de capillaire obtenue avec clarity imaging et fluoresence. D : vesseauw sanguins, myocyte et collagène obtenue avec un microscope à fluoresence. WKY : Wistar Kyoto (rat transgénique). WGA : Wheat Germ Agglutin (utilsé pour marquer la membrane cellulaire). SHG : Second Harmonic generation signal.  Adaptée de \cite{Olianti_2020}}
  \label{fig:clarity}
\end{figure}

Clarity Imaging permet d’obtenir différents contrastes d’images (vaisseaux sanguins, collagène, cellules) mais les contrastes dépendent de la bonne préparation de l’ échantillon. De plus, la taille de l’échantillon est réduite car il doit pouvoir être dans suffisamment petit pour être dans le microscope. Grâce à des algorithmes de tenseur de structure, la structure orthotrope du myocarde peut être retrouvé.
\clearpage
\subsubsection{Micro Computed Tomography et Syncrotron}

La TDM peut aussi être utilisé sur des cœurs ex vivo. Les différents tissus ayant des radio-opacités (absorptions des rayon X) différentes, il est possible d’obtenir un volume avec un contraste en niveau de gris correspond à cette radio-opacité. Les scanners de micro computed tomography (MicroCT) possèdent un tunnel large (50 – 100 mm) leurs permettant d’imager de plus large échantillon comme des cœurs entiers de mammifères. 

Le Syncrotron (accélérateur de particules) est une vaste source de rayon X (dose 10 milliards de fois supérieur à la dose hospitalière), c’est pourquoi, certains scientifiques utilisent le Synclotron pour avoir un volume anatomique (en niveau de gris) avec une résolution isotope proche 1 $\mu$m \cite{Planinc2021}, contre 20-40 $\mu$m pour le MicroCT classique. 

\begin{figure}[!htbp]
  \begin{center}
    \includegraphics[width=0.9\textwidth]{Chapitre1/microCT.png}
     \end{center}
    \caption{cœur exvivo imagé par Rayons-X. Gauche : cœur de rat imagé avec un Syncrotron avec une résolution inférieure à 1 um. La structure laminaire est visible en gris et la MEC est plus radio-opaque donc en gris foncé. Adaptée de \cite{Planinc2021}. Droite : Vue en LA d’un cœur de cochon séché à une résolution de 20 $\mu$m. Les fibres de Purkinje dans le VG et les tendons entre les valves et les piliers sont visibles \cite{Pallares_Lupon_2022}}
  \label{fig:uCT}
\end{figure}

Les tissus mous (composé en majorité d’eau) ont un faible poids moléculaire en comparaison avec les os. Les tissus mous vont alors provoquer une faible atténuation des rayons X et et un faible contraste au MicroCT. Pour résoudre ce problème, une nouvelle méthode de préparation des échantillons a été développer dans notre laboratoire. Les cœurs sont déshydraté avec un lavage à l’éthanol puis perfusé avec de l’éthanol et de l’ hexaméthyldisilazane (HMDS) et ensuite juste avec de l’HMDS, pour finir un séchage lent à l’air libre pendant plusieurs jours \cite{Pallares_Lupon_2022}. 

Grâce au séchage, la structure laminaire du myocarde est visible car les plans du myocarde vont s’écarter légèrement. Comme pour le \textit{clarity imaging}, algorithmes de tenseur de structure peuvent être appliqué sur les volumes pour obtenir la structure laminaire du cœur. 

	\subsubsection{L’Imagerie en Tenseur de structure}

Les techniques d’imagerie présenté jusqu’ici sont permettent de voir la structure du myocarde, et de différencier les différentes composantes du myocarde (myocytes, MEC, capillaires…). L’orientation des chaînes de myocyte n’est pas une information directement disponible à partir de ces volumes. Pour obtenir l’orientation des myofibres, il est utilisé un objet mathématique appelé tenseur. Plus de détails sur l’obtention de l’orientation d’un vecteur à partir d’un tenseur est détaillé Partie 3. Pour obtenir un tenseur dit de structure (ST) d’une image I, structure pour éviter de le confondre avec un tenseur de diffusion, les gradients $\Delta$ dans les directions x,y et z sont calculés puis le tenseur est défini comme suit : 


\begin{equation}
\nonumber
ST = 
\begin{bmatrix}
\Delta I_x \cdot \Delta I_x & \Delta I_x \cdot \Delta I_y & \Delta I_x \cdot \Delta I_z\\
\Delta I_y \cdot \Delta I_x & \Delta I_y \cdot \Delta I_y & \Delta I_y \cdot \Delta I_z\\
\Delta I_z \cdot \Delta I_x & \Delta I_z \cdot \Delta I_y & \Delta I_z \cdot \Delta I_z\\
\end{bmatrix}
\end{equation}

\subsubsection{L’IRM ex vivo à haut champ magnétique}

Le champ magnétique utilisé en clinique est dit champ moyen (entre 0.55 T et 3 T), il respecte des normes d’exposition aux ondes électromagnétiques. L’intensité du signal RMN est dépendante de la valeur locale d’aimantation. En augmentant en champ magnétique, le rapport signal sur bruit (signal to noise ratio, SNR) augmente et couplé à des gradients avec de plus grandes amplitudes, la résolution spatiale devient plus fine. Le champ magnétique est généré par un aimant supraconducteur (résistance électrique très faible) qui devient supraconducteur à très faible température (< 269°) en étant plongé dans un bain d’hélium liquide. L'augmentation du champ magnétique raccourcit généralement les temps de relaxation T1 et T2 car elle intensifie les interactions spin-réseau pour T1et les inhomogénéités de champ pour T2, conduisant à une relaxation plus rapide.  Réduire les T1 et T2 ne semble pas être intéressant à première vue mais le gain de SNR permet d’obtenir une bonne pondération T1 ou T2 (petite digression, à contrario, baisser en champ magnétique rallonge les temps T1 et T2, permettent aussi un bon contraste en pondération T1 ou T2, certains constructeurs de scanner IRM se base sur ce postulat pour proposer une nouvelle gamme de scanner dit bas champ (0.55 T)).  
L’IRM pondérée en diffusion profite beaucoup de ce gain en SNR permettant d’augmenter la valeur b sans dégrader l’intensité de l’écho RMN, révélant une structure plus encore plus précise qu’en imagerie de diffusion \textit{in vivo}. 

Les noyaux atomiques précessent à une fréquence fixée par le champ magnétique, la fréquence de Larmor donnée par l’équation de Larmor : $\omega_0=\gamma \cdot B \cdot (1-\sigma)$ avec $\sigma$ le blindage chimique (les électrons génèrent un champ magnétique locale, créant un blindage au champ de l’aimant supraconducteur). Chaque molécule possède un nombre différent d’atomes d’hydrogène, chaque fréquence de Larmor est associée à une molécule. En augmentant le champ magnétique, l’écart entre les fréquences de Larmor de deux molécules différentes augmente.  La spectroscopie RMN est la capacité de différencier plusieurs molécules sur un spectre RMN. 

La graisse possède un environnement magnétique différents de la molécule d’eau (chaine carbonée), par approximation, la graisse precesse a une fréquence différente de celle de la molécule d’eau. Le déplacement chimique est la différence relative de fréquence entre une molécule et une molécule de référence (l’eau, monoxyde de dihydrogène, H20). A 1.5T, le déplacement chimique entre l’eau la graisse est de 220 Hz alors qu’9.4 T il est de 398 MHz (déplacement de 3.5 ppm). Grâce à cette plus grande différence, le contraste de la graisse sera largement atténué par rapport au signal du tissu (constitue d’eau). La méthode IDEAL \cite{Haliot2021} utilise ce postulat pour acquérir plusieurs échos et séparer les signaux de la graisse et de l’eau. Comme pour la diffusion, l’imagerie de la graisse profite largement de la hausse du champ magnétique. 

L’étude du collagène est une thématique centrale dans l’étude de la structure cardiaque. Plusieurs éléments comme la fibrose ou les fibres de Purkinje en sont massivement constitués (Paragraphe myocarde). Comme pour les autres modalités d’imagerie, l’IRM permet de distinguer le collagène des autres composantes du myocarde. Une technique populaire en neurologie pour distinguer les gaines de myélines (isolant les neurones) est le transfert d’aimantation ou magnétisation (MT). Le principe est basé sur l'application d'une impulsion de saturation à une population de spins, généralement des protons liés à des macromolécules comme le collagène. Cette saturation est ensuite transférée à l'eau libre par échange magnétique, ce qui entraîne une diminution de l'intensité du signal de l'eau observable en RMN. Les résultats du MR sont souvent représentés sous la forme d’un ratio (MTR) entre un volume avec le module de préparation du MT  ($S_{MT}$) et un volume sans ($S_0$). 
\begin{equation}
\nonumber
MTR=\ \frac{S_0-\ S_{MT}}{S_0}
\end{equation}

ou \cite{Haliot2021} :

\begin{equation}
\nonumber
MTR=\ \frac{S_{MT}}{S_0} 
\end{equation}


\begin{figure}[!htbp]
  \begin{center}
    \includegraphics[width=0.9\textwidth]{Chapitre1/MT_kylian.png}
     \end{center}
    \caption{Imagerie de la graisse avec une séquence ideal (c-d) et MTR (f) sur un cœur avec une dysplasie arythmogène du ventricule droit. Les résultats du de la séquence ideal sont exprimé sous la forme d’un ratio en densité de proton appliqué à la graisse. Adaptée de [10]}
  \label{fig:IRM_exvivo_MT}
\end{figure}

L’IRM ex vivo à haut champ magnétique a de nombreux avantages mais il nécessite une instrumentation adaptée et plus couteuse. Il permet néanmoins un large panel de contraste sans détruite l’échantillon. Les résolutions en IRM à haut champ sont variables, comme dit précédemment car elles dépendent des gradients mais pour des cœurs entiers de gros mammifères. Il est possible d’obtenir en fonction du contraste souhaité et du temps d’acquisition disponible (une plus grande résolution implique plus de temps d’acquisition) une résolution allant de 100 $\mu m$ a 800 $\mu m$ isotropique.
%Pour montrer l'environnement bulletList

\subsubsection{Conclusion}

Le cœur est un organe en mouvement permanent, et l’architecture laminaire change en fonction de la diastole et de la systole. Établir une loi sur l’orientation des myocytes sur un cœur ex vivo quelques soit la modalité utilisée n’est pas chose aisé. Certaines équipes\cite{NiellesVallespin2019} \cite{STREETER1969} arrêtent un cœur en diastole puis un autre en systole et, après imagerie, en déduise l’orientation dans les deux cas. Dans la majorité des cas, les cœurs sont arrêtés dans une phase systolique.

En conclusion, l’IRM de diffusion est un outil idéal pour étude l’orthotropie du cœur de grand mammifères (humain, brebis ou cochon) c’est pourquoi il est une méthode d’imagerie incontournable dans ce domaine. La partie Trois de ce chapitre d’introduction expliquera le foncionnement de l’IRM de diffusion et de ces métriques associées. 



\subsection{La diffusion dans un organe spécifique, le cœur}

\subsubsection{Décrire le tissu cardiaque à l’aide du DTI}

Jusqu’ici, nous avons vu la diffusion dans son ensemble sans savoir si le tissu étudié été du muscle fibro-squelettique, de la matière blanche du cerveau ou tout autre tissu biologique. Le paragraphe suivant est un résumé de la diffusion dans le muscle cardiaque avec un inventaire des différentes orientations connus ainsi que de métriques dérivés du DT utilisé en IRM de diffusion cardiaque.

Nous avons vu que le myocarde était constitué de cellules cardiaques appelés cardiomyocytes qui forment un système complexe de fibres musculaires laminaires (sous la forme de feuillets). Le tissu cardiaque est extrêmement perfusé (capillaires), un modèle de signal de diffusion existe pour mieux décrire la diffusion en fonction de la perfusion \cite{NiellesVallespin2019} : 
\begin{equation}
\nonumber
\frac{S}{S_0}=fe^{-b\left(D+D\ast\right)}+\left(1-f\right)e^{-bD}
\end{equation}

Avec f (sans dimension) la fraction de perfusion (le pourcentage du volume occupé par les capillaires). D* est appelé le coefficient de pseudo-perfusion reflétant le déphasage induit par les capillaires. 

Dans le cas d’acquisitions ex vivo, le tissu cardiaque n’est plus perfusé donc ce modèle n’est plus valide mais il le reste dans le cadre d’acquisition \textit{in vivo}.

Les structures microscopiques (membranes cellulaires, matrice extra-cellulaire/espace interstitielle ...) restreignent la diffusion des molécules d’eau. L’orientation principales des cardiomyocytes sera donnée par le premier vecteur propre $e_1$, le deuxième $e_2$ donne l’orientation des feuillets (un plan) et le troisième $e_3$ est la normale du plan des feuillets (Figure \ref{fig:dti_heart_plan}). 

\begin{figure}[!h]
\begin{center}
  \includegraphics[width=0.9\textwidth]{Chapitre1/dti_heart_plan.png}
   \end{center}
  \caption{Vecteurs propres issus de la diffusion dans le tissu cardiaque. A : schéma de l’organisation en feuillets de l’endocarde à l’épicarde. B-D : $e_1$,$e_2$ et $e_3$ dans le VG d’un cœur humain,respectivement}
\label{fig:dti_heart_plan}
\end{figure}

Les métriques comme la FA et l’ADC sont en moyenne égale à 0.30 et 1.5 x $10^{-3}Ò$ $mm^2/s$ respectivement. Si la perfusion, n’est pas compensée dans le calcul des métriques de diffusion, l’ADC peut-être surestimé voir supérieur à la limite théorique qu’est la diffusion de l’eau dans un milieu non contraignant (2.9 x $10^{-3}$ $mm^2/s$) \cite{Moulin2023}. 

\subsubsection{Les métriques dérivées des vecteurs propres du DT} 


L’orientation des feuillets s’étend dans la direction des cardiomyocytes avec une variation helicoïdale entre l’endocarde et l’épicarde et oblique par rapport au plan tangent de l’épicarde (Figure \ref{fig:model_fibre}). 

L’Angle Helix (HA) est une métrique dérivée de $e_1$ permettant d’obtenir une représentation scalaire de l’orientation des myocytes \cite{Ferreira2013}. Il est obtenu en projetant $e_1$ dans un plan tangentiel au myocarde (épicarde) puis en calculant l’angle sous-tendu avec le plan en petit axe.

L’indice de désordre du myocarde (\textit{Myocardium Disarray Index} ,MDI) est un métrique aussi dérivée de e1 reflétant la désorganisation locale des cardiomyocytes \cite{Wu2004} \cite{GarciaCanadilla2019}. Le MDI quantifie pour chaque voxel l’uniformité de la direction principale des myocytes au voisinage de ce même voxel. Il varie entre 0 et 1 avec 1 signifiant aucune différence d’orientation des cardiomyocytes ($e_1$ équivalent pour tous les voxels) et 0 le contraire ($e_1$ tous différents au voisinage proche du voxel). 
\begin{figure}[!h]
\begin{center}
  \includegraphics[width=0.9\textwidth]{Chapitre1/mdi_HA.png}
   \end{center}
  \caption{Schémas représentant les différentes étapes de calcul permettant d’obtenir l’HA (droite) et le MDI (gauche). Adaptée de \cite{Ferreira2013}}
\label{fig:dti_HA_mdi}
\end{figure}

L’angle formé par les feuillets (E2A) est calculé à l’aide du vecteur $e_3$. Il est l’angle sous-tendu entre la projection du vecteur $e_3$ dans le plan radial ayant pour origine le centre du VG et le plan tangentiel à l’épicarde\cite{Dou2002}. Dans la littérature, il peut être affiché en valeur absolue. 

\begin{figure}[!h]
\begin{center}
  \includegraphics[width=0.9\textwidth]{Chapitre1/ha_e2a_figure.png}
   \end{center}
  \caption{Imagerie pondérée en diffusion in vivo et métriques dérivés des vecteurs propres (HA et E2A) chez un volontaire sain. Adaptée de \cite{Ferreira2014}}
\label{fig:dti_e2A}
\end{figure}


\section{Conclusion}

Dans ce chapitre d'introduction, nous avons vu que le myocarde est un tissu hétérogène mais structuré, constitué de cardiomyocytes agrégées sous la forme de chaînes (par abus de langage, les myofibres), ces chaînes formant des plans. L'orientation des myofibres est importante car la propagation du PA, nécessaire à l'excitabilité et à la contraction des cellules (permettant le mouvement de contraction du cœur), dépend de celle-ci. Un désordre trop important est l'une des causes des maladies du rythme cardiaque. Il existe différentes méthodes pour observer l'orientation de ces fibres, mais l'IRM de diffusion est un \textit{gold standard} en matière de recherche, c'est pourquoi nous l'avons utilisée dans cette thèse. Nous avons également voulu approfondir l'étude de l'anisotropie du myocarde en utilisant un modèle avancé de DTI. 



 %begin{bulletList}
 %\item Premier point
% \item Deuxième point
% \item Et un acronyme utilisé (défini dans Acronymes.tex) \ac{DTI}
%\end{bulletList}
