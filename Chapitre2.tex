%!TEX root = Manuscrit.tex
\chapter{Description multimodale quantitative et qualitative du point d’insertion postérieur du ventricule droit chez l’homme et la brebis}
\label{chap:RVIP}

%\localtableofcontents
\minitoc
\section{Préface}

Ce chapitre est consacré à la présentation des travaux réalisés sur l’étude de l’orientation des cardiomyocytes du point d’insertion postérieur du ventricule droit (RVIP).

Ces travaux de thèse ont fait l’objet d’une publication scientifique dans le journal Journal of Cardiovascular Magnetic Resonance en décembre 2023 et est retranscrite à la suite du manuscrit et il est également disponible à ce lien ( \href{https://doi.org/10.1186/s12968-023-00989-y}{lien}). 

L’objet de ce chapitre est de présenter le contexte du projet et les objectifs. Les travaux méthodologiques et les résultats sur lesquels j’ai travaillé vont être présentés. J’invite néanmoins le lecteur à ce reporter à l’article pour avoir une vision plus complète des résultats et de discussion associée. 
 \section{Introduction}
 
 Les cardiomyocytes sont organisés sous la forme de chaîne de myocytes (Chapitre \ref{chap:intro}), par approximation, ces chaînes sont appelés fibres ou myofibres. Dans le ventricule gauche (VG), l’orientation des fibres varie de manière hélicoïdale entre l’épicarde et l’endocarde. L’angle formé entre cette orientation et l’axe base-apex varie linéairement entre – 60° et 60° de l’épicarde à l’endocarde. Dans le ventricule droit (VD), la variation transmurale de l’orientation des fibres est aussi linéaire avec une variation de -25° à 90°. 
\\
Les points d’insertion des ventricules droit et gauche sont représentés par deux carrés verts sur la Figure \ref{fig:repere_RVIP}, un pour le côté antérieur et un pour le côté postérieur. 

\begin{figure}[!ht]
  \begin{center}
    \includegraphics[width=0.8\textwidth]{Chapitre2/figure_RVIP_v2_repere.png}
  \end{center}
  \caption{Vue en petit axe d’un cœur humain en IRM clinique, les points d’insertion sont visualisés par des carrés verts, adaptée de \cite{Li2020}}
  \label{fig:repere_RVIP}
\end{figure}

Une description précise, c’est-à-dire tridimensionnelle et à haute résolution ($<1 mm^3$) de l’architecture cardiaque du RVIP chez les grands mammifères est relativement incomplète si ce n’est absente de la littérature. Cette zone du myocarde est connue pour être plus complexe que le ventricule gauche dû à l’intersection des cavités ventriculaires. D’un point de vue pathologique, un rehaussement tardif au gadolinium est présent dans cette zone chez des patients atteints d’hypertension artérielle et pulmonaire (HTAP) (Figure \ref{fig:gado_RVIP}.I) \cite{Swift2014} et chez des athlètes de haut niveau \cite{DomenechXimenos2020} (Figure \ref{fig:repere_RVIP}.II).
\\
\begin{figure}[!ht]
  \begin{center}
    \includegraphics[width=0.8\textwidth]{Chapitre2/figure_lge_rvip.png}
  \end{center}
  \caption{Rehaussement tardif chez des patients HTAP (I, \cite{Swift2014}) et chez un athlète (II, \cite{DomenechXimenos2020})}
  \label{fig:gado_RVIP}
\end{figure}

Une récente étude \cite{Friedberg2013} a étudié l’impact du remodelage structurel chez le lapin HTAP en utilisant de l’échographie et de l’histologie cependant,  cette étude reste sur le petit animal et le remodelage structurel a été étudié dans le VG et le VD mais pas le RVIP. (Figure \ref{fig:histo_smerup}.I). Un remplacement du myocarde sain par de la fibrose a été observé dans le ventricule droit et gauche.

\begin{figure}[!ht]
  \begin{center}
    \includegraphics[width=0.8\textwidth]{Chapitre2/figure_friedberg_smerup.png}
  \end{center}
  \caption{coupe d’histologie de cœurs de lapin sham et HTAP \cite{Friedberg2013} (collagène en rouge, myocarde sain en beige). II : tractrographie et statistique transmural de différentes régions du ventricule gauche.\cite{Smerup2008}}
  \label{fig:histo_smerup}
\end{figure}


L’étude locale de l’orientation des cardiomyocytes par IRM de diffusion ex vivo est une technique d’imagerie existante depuis de nombreuses années. En 2009, une étude a imagée des cœurs ex vivo de cochons à une résolution isotrope de 1.3 mm à partir d’une séquence 2D multi-coupe \cite{Smerup2008}.  Cette étude de Smerup et al insiste sur la possibilité d’une connectivité longue distance entre différentes parties des ventricules droit et gauche mais elle ne s’intéresse pas de manière locale à l’orientation des chaines de cardiomyocyte dans le RVIP (Figure \ref{fig:histo_smerup}.II).
 \\
Une première étude a eu lieu sur trois cœurs de brebis \textit{ex vivo} par J. Magat et V. Ozenne \cite{Magat2022} en utilisant l’IRM de diffusion à 9.4 T. Une structure macroscopique avec des cardiomyocytes ayant une orientation base-apex a été révélé avec une architecture de myofibres reproductible entre les trois échantillons (Figure \ref{fig:MAGAT}).  Cette étude établira les fondements des travaux de ce chapitre ; néanmoins, elle se concentre sur un seul type de gros mammifère (brebis) et il n’y a aucune validation de cette architecture par une méthode \textit{gold standard} (histologie).

\begin{figure}[!ht]
  \begin{center}
    \includegraphics[width=0.5\textwidth]{Chapitre2/journal.pone.0271279.g004.PNG}
  \end{center}
  \caption{Visualisation du RVIP dans la zone basale à l'aide de modèles anatomiques, de l'FA et du tenseur de diffusion en vue sagittale, coronale et transversale à l’aide d ‘un template de (N=3) cœurs exvivo de brebis . (A) Vue large et (B) vue zoom du modèle anatomique sur le RVIP. (C) FA. (D) E1 superposées aux données anatomiques. Adaptée de \cite{Magat2022}}
  \label{fig:MAGAT}
\end{figure}

\section{Objectifs}

Partant de la description de Magat du RVIP \cite{Magat2022},  le but de cette étude est de vérifier la reproductibilité inter-espèce de cette structure macroscopique avec l’intégration de données humaine. Pour cela, le nombre de cœur ovins est passé de 3 à 5 et 5 cœurs humains issus du programme Cadence ont été étudié. 
\\
D’autres modalités d’imagerie ont été utilisés pour valider cette nouvelle structure : de l’histologie pour valider en utilisant un \textit{gold standard} et une méthode plus récente (donc sujette à controverse), le MicroCT sur cœur entier séché \cite{Pallares_Lupon_2022} pour obtenir une imagerie anatomique en 3D avec une résolution isotrope de 20 $\mu m$.
\\
Les diverses propriétés électrophysiologiques du myocarde sain, de la cicatrice dense et de la zone frontalière créent des zones d'hétérogénéité fonctionnelle, conduisant à des réentrées. Cette hétérogénéité peut affecter le comportement normal de la restitution de la durée du potentiel d'action (PA) des myocytes. Les variations spatiales du PA et les alternances de vitesse de conduction peuvent favoriser l'émergence d'un mécanisme de réentrée \cite{MejaLopez2019}, entraînant des arythmies. Pour étudier l’impact de la nouvelle architecture de myofibre sur les propriétés électrophysiologie, des simulations de propagation de potentiels électriques ont été faites avec différents scénarios d’architecture et de présence ou non de fibrose (conduction imposée nulle à l’interface de la structure base-apex : triangle bleu sur la Figure \ref{fig:MAGAT}).

\section{Matériels et Méthodes}
Les cœurs \textit{ex vivo} (Table \ref{tab:table1}) ont été fixés à partir d’une solution de formalin (10 $\%$) et d’un agent de contraste à base de gadolinium (0.2 $\%$). Les cœurs ovins ont été canulés à partir de l’aorte et les cœurs humains ont été canulés de manière retro-coronarienne. Enfin, les cœurs sont plongés dans la solution puis perfusés pendant 24 heures.	
\begin{table}[!ht]
\large
\begin{tabular}{cccc}
\hline
\textbf{Heart n°}  &  \textbf{Age [y]} & \textbf{Sex} & \textbf{dimensions [cm x cm x cm]} \\
\hline
\hline
\#H1 &   53 &    F &   10.9 x 8.0 x 14.1 \\
\#H2  &   56 &    M &    8.6 x 9.4 x 10.7  \\
\#H3 &   82 &    F &   8.2 x 10.1 x 11.1 \\
\#H4 &   83 &    F &  10.1 x 8.1 x 11.4   \\
\#H5 & 83& F & 8.4 x 7.4 x 12.1\\ 
\hline
\hline
\#S1 &   2 &    F &  7.8 x 5.3 x 10.5  \\
\#S2  &  2 & F & 7.6 x 5.3 x 10.0  \\
\#S3 & 2 & F& 7.5 x 5.1 x 9.9 \\
\#S4 &  2 & F & 8.4 x 6.3 x 9.8\\
\#S5 & 2& F& 8.2 x 6.3 x 10.6\\ 
\#S6 & 10& F & 6.3 x 5.2 x 6.2\\ 

\hline

\end{tabular}
\label{tab:table1}
\caption{Caractéristiques principales des (N =11) coeurs \textit{ex vivo} utilisés dans l'étude, le coeur \# S6 correspond à l'échantillion MicroCT. S pour sheep (brebis) et H pour human (humain).}
\end{table}

Ensuite, les cœurs ex vivo (\# S1-5 ; \# H1-5) ont été scannés à l’IRM Bruker Biospec 9.4 T avec une antenne 7 éléments Tx/Rx de 20 cm de diamètre et des gradients de 300 mT/m.  L’utilisation d’un IRM à haut champ magnétique permet d’avoir un SNR élevé ainsi que d’avoir des gradients puissants pour augmenter la résolution spatiale. 
\\
Une séquence d’écho de gradient avec un angle de bascule faible (TR/TE/$\alpha$ = 30/9 ms/ 21$°$) dites FLASH (\textit{Fast Low Angle Single sHot}) a permis d’obtenir un contraste pondéré T1 et pondéré T2* avec une résolution de 150 $\mu m$ pour un temps d’acquisition total de 18h (8 moyennes). Cette image servira de repère anatomique et permettra d’évaluer la présence ou non de défaut sur l’image ou la présence d’éventuelles vaisseaux. 
\\
Les données de diffusion (DWI) ont été acquises à partir d’une séquence d’écho de spin (TR/TE = 500/22 ms) avec 6 directions non-colinéaire de diffusion ( b = 1000 $s/mm^2$, 1 moyenne par directions). Une image non pondérée en diffusion (b = 0 $s/mm^2$, 3 moyennes) a été acquise en plus pour un temps total d’acquisition de 24h et une résolution de 600 $\mu m$ isotrope.
\\
Après débruitage et correction de biais dû aux hétérogénéités du champ$ B_1$, un tenseur de diffusion (DT) a été calculé avec le logiciel Mrtrix3 à partir des DWI. Pour faciliter l’analyse futur des DT, une transformation affine a été appliquée à tous les échantillons pour redresser les images et obtenir les cœurs orientés de manière parallèle au grand axe. Cette réorientation a été faite avec le logiciel ANTs et selon la méthode du Log-Euclidien décrite par \cite{Arsigny2006}.  
\\
Les cartes d’ADC et de FA ainsi que les vecteurs propres $e_1$,$e_2$,$e_3$ ont été extraits du DT dans ce nouvel espace. Une tractrographie déterministe \cite{Mori1999} a été calculée à partir d’e1. Cette tractrographie fut ensuite filtrée avec une orientation en fonction de l’axe z . Cette méthode permet d’obtenir une nouvelle tractrographie qui ne garde que les orientations de myofibres ayant une orientation base-apex. L’étape suivante consiste à compter le nombre de fibres mathématiques issus tractographie dans une grille cartésienne (résolution et taille identique aux cartes dérivées du DT) et de faire un seuillage pour obtenir un masque binaire des myofibres ayant une orientation base-apex. Une dernière étape consistant à délimiter notre zone d’intérêt (manuellement) pour ne garder que le faisceau de fibre (Figure \ref{fig:pipeline}).
La position du faisceau par rapport aux cavités a été calculés avec le logiciel 3DSlicer.

\begin{figure}[!ht]
  \begin{center}
    \includegraphics[width=0.95\textwidth]{Chapitre2/figure_pipeline_rvip.png}
  \end{center}
  \caption{Chaîne de traitement des données de diffusion pour obtenir une segmentation d'un faisceau de fibre en orientation base-apex. Flèches bleues : traitement automatique; Flèches rouges : traitement manuel}
  \label{fig:pipeline}
\end{figure}

Dans le cadre de cette étude, des simulations d’électrophysiologie ont été effectuées en propageant un signal électrique sur un maillage généré à partir de l’orientation de fibres. Trois scénarios d’orientation de fibres ont été envisagés : le premier avec les données dites « rules-bases », inspiré des travaux de \cite{Bayer2012}, ensuite avec les données expérimentales sur un cœur humain et le dernier scénario avec ces mêmes données expérimentales mais avec conduction nulle imposée à l’interface entre le faisceau et le reste du myocarde. 
\\
L’IRM cardiaque de diffusion \textit{in vivo} étant une piste prometteuse pour l’étude de la structure du myocarde chez les patients sans agent de contraste, de nouvelles acquisitions (en 2D cette fois ci) \textit{ex vivo} sur le cœur de premier $\#$ S1 ont été faites pour étudier l’impact de la valeur b (de 100 à 1000 $s/mm^2$) et un rééchantillonnage des tenseurs $\#$ H1-5 avec des résolutions utilisées en diffusion cardiaque in vivo (2x2x5 mm ou 2x2x8 mm).
\clearpage
\section{Résultats}

Le faisceau de cardiomyocyte est présent sur N = 9 (90 $ \% $) cœurs de brebis et humain. Le volume moyen pour tous les cœurs est de 0.35 $\pm$ 0.15 $cm^3$ , la taille en grand axe est de 23.0 $\pm$ 5.6 mm et elle est située en moyenne plus proche de la cavité de VD. La FA moyenne dans le volume du faisceau est de 0.29 $\pm $ 0.06 pour les cœurs ovins et 0.27 $\pm$ 0.05 pour les cœurs humains, en comparaison, 0.26 $\pm$ 0.09 et 0.24 $\pm$ 0.10 dans le cœur entier de brebis et humain, respectivement. (Figure \ref{fig:FA}).

\begin{figure}[!ht]
  \begin{center}
    \includegraphics[width=0.95\textwidth]{Chapitre2/fa.png}
  \end{center}
  \caption{Distribution de la FA dans tous les échantillions IRM \textit{ex vivo}. Rouge : histogramme de la FA dans le coeur entier. Bleu : histogramme de la FA dans la segmentation du faisceau de fibres}
  \label{fig:FA}
\end{figure}


Figure \ref{fig:diff_reso} montre les résultats de la valeur b (Figure \ref{fig:diff_reso}.I) et du rééchantillonnage sur le $e_1$ (Figure \ref{fig:diff_reso}.II). Le faisceau de fibres est visible à partir de b = 500 $s/mm^2$ et reste visible malgré la faible résolution (résolution utilisée en imagerie cardiaque de diffusion). Le signal pondéré en diffusion reste suffisant pour avoir des vecteurs propres définis.

\begin{figure}[!ht]
  \begin{center}
    \includegraphics[width=0.95\textwidth]{Chapitre2/figure_b_values_et_reso.png}
  \end{center}
  \caption{: Effet de la valeur de b et de la résolution sur la visualisation du faisceau de fibres dans le RVIP sur cœur de brebis. I : moyenne des images pondérées en diffusion de chaque direction. L’intensité de l’image diminue avec l’augmentation de la valeur de b en accord avec la littérature. Évolution d’$e_1$ en fonction de la valeur de b, le faisceau de fibres est visible à partir de b = 500  $s/mm^2$. II : rééchantillonnage des données de diffusion ($\#$ H1) à une résolution \textit{in vivo}}
  \label{fig:diff_reso}
\end{figure}
%
\clearpage
\section{Discussion}

Cette étude a permis de mieux comprendre l’architecture local des agrégats de cardiomyocytes dans le RVIP. L’automatisation du processing a permis d’obtenir des résultats de manière reproductible et indépendante d’un opérateur. La validation par histologie a permis de confirmer nos données de diffusion. 

Les travaux sur la résolution et l’impact de la valeur de b ont permis d’envisager de visualiser le l’architecture précise du RVIP in vivo. Une étude sur un modèle préclinique de cochon avec HTAP était en cours et nous avons pu tester une séquence de diffusion cardiaque in vivo sur ces cochons. Cela nous a permis de nous familiariser avec la séquence créée par K. Moulin et de pouvoir axer la suite de mes travaux de recherche sur la diffusion cardiaque in vivo. Malheureusement, ces tests ont été effectué sur un Aera 1.5 T de chez Siemens et comme cela a été décrit dans la littérature de nombreuses fois, 1.5 T ne permet d’avoir un rapport signal sur bruit suffisant pour obtenir un tenseur de diffusion correct.  

\section{Travaux effectués}

Ces travaux de thèse ont fait l’objet d’une publication scientifique dans le journal \textit{Journal of Cardiovascular Magnetic Resonance} en décembre 2023 et est retranscrite à la suite du paragraphe introductif précédent (\href{https://doi.org/10.1186/s12968-023-00989-y}{lien} ).

 %begin{bulletList}
 %\item Premier point
% \item Deuxième point
% \item Et un acronyme utilisé (défini dans Acronymes.tex) \ac{DTI}
%\end{bulletList}
